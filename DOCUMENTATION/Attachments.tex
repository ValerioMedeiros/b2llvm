
\section*{Attachments}

\subsection{Structure examples}

The following code example is defining a structure:\\
\begin{pascalcode}
MACHINE Rec
SETS
 ID={aa,bb}
VARIABLES xx, bank, total
INVARIANT
 xx : struct(name : ID, balance : NAT) & bank : struct(name : ID) &
 total:struct(money:NAT)
INITIALISATION 
 bank := rec(name:bb) ||
 xx:= rec(name:aa, balance:10) ||
 total := rec(money:20)
OPERATIONS

  res<--positive = 
  PRE xx'balance > 0 THEN
  res:=TRUE
  END;

  withrdaw(amt) = 
  PRE amt:NAT1 & (xx'balance >= amt) THEN
     xx := rec(name:xx'name, balance:(xx'balance - amt))
  END;
  
  unsafe_dec = 
  BEGIN
       xx'balance := xx'balance - 1
  END
END
\end{pascalcode}


\begin{pascalcode}
IMPLEMENTATION
    Rec_i

REFINES
    Rec

CONCRETE_VARIABLES
   xx ,
   bank ,
   total

INITIALISATION
   
   bank := rec(name:bb) ;
   xx := rec(name:aa, balance:10) ;
   total :=rec(money:20)
OPERATIONS
  res<--positive = 
  IF xx'balance > 0 THEN
       res:= TRUE 
  ELSE res := FALSE
  END;

  withrdaw(amt) = IF   amt >=0 & amt <= MAXINT & (xx'balance >= amt) THEN 
     xx := rec(name:xx'name, balance:(xx'balance - amt))
  END;

  unsafe_dec = BEGIN
       xx'balance := xx'balance -1 
  END
END
\end{pascalcode}




\subsection{Arrays examples}

A generic example available at repository ProB\footnote{
\url{https://github.com/bendisposto/probparsers/blob/develop/bparser/src/test/resources/parsable/Array.mch}}. 
This example can be used in approach using imported components.


\subsection{A  simple example and its LLVM}

\begin{pascalcode}
MACHINE
   Array
VARIABLES arr

INVARIANT 
    arr : ((0..99) --> 0..1000)

INITIALISATION arr := {1|->100}

OPERATIONS 

set(ix,tt)=
        PRE ix : (0..99) & tt : (0..1000)
        THEN arr := arr <+ {ix |-> tt}
        END ;

tt <-- read(ix) =
        PRE ix : (0..99)
        THEN tt := arr(ix)
        END;

swap(ix, jx)=
        PRE ix : (0..99) & jx : (0..99)
        THEN arr := arr <+ {ix |-> arr(jx), jx|-> arr(ix)}
        END

END
\end{pascalcode}

\begin{pascalcode}
IMPLEMENTATION 
   Array_i
REFINES
   Array
CONCRETE_VARIABLES
   arr
INVARIANT 
    arr :  ((0..99) --> 0..1000)
INITIALISATION
   arr(1) := 100
OPERATIONS
   set ( ix , tt ) =
   IF
       ix >= 0 & ix <=99 & tt >= 0  & tt <= 1000
   THEN
       
      arr(ix) := tt 
   END;

   tt <-- read ( ix ) =
   IF
       ix >= 0 & ix <=99
   THEN
       tt := arr ( ix )
   ELSE
       tt := 0
   END;

   swap ( ix , jx ) =
   IF
        ix >= 0 & ix <=99 & jx >= 0 & jx <=99 
   THEN

       VAR temp IN
           temp := arr(jx);
           arr(jx) := arr(ix);
           arr(ix) := temp
       END
   END
END
\end{pascalcode}

\begin{pascalcode}
@Array__arr = internal global [100 x i32] zeroinitializer, align 16

define void @Array__INITIALISATION() nounwind ssp {
  store i32 100, i32* getelementptr  ([100 x i32]* @Array__arr, i32 0, i64 1)
  ret void
}

define void @Array__set(i32 %ix, i32 %tt) nounwind ssp {
  %1 = alloca i32
  %2 = alloca i32
  store i32 %ix, i32* %1
  store i32 %tt, i32* %2
  %3 = load i32* %1
  %4 = icmp sge i32 %3, 0
  br i1 %4, label %5, label %19

; <label>:5                                       ; preds = %0
  %6 = load i32* %1
  %7 = icmp sle i32 %6, 99
  br i1 %7, label %8, label %19

; <label>:8                                       ; preds = %5
  %9 = load i32* %2
  %10 = icmp sge i32 %9, 0
  br i1 %10, label %11, label %19

; <label>:11                                      ; preds = %8
  %12 = load i32* %2
  %13 = icmp sle i32 %12, 1000
  br i1 %13, label %14, label %19

; <label>:14                                      ; preds = %11
  %15 = load i32* %2
  %16 = load i32* %1
  %17 = sext i32 %16 to i64
  %18 = getelementptr  [100 x i32]* @Array__arr, i32 0, i64 %17
  store i32 %15, i32* %18
  br label %19
; <label>:19                                      ; preds = %14, %11, %8, %5, %0
  ret void
}
\end{pascalcode}


\begin{llvmcode}
define void @Array__read(i32 %ix, i32* %tt) nounwind ssp {
  %1 = alloca i32, align 4
  %2 = alloca i32*, align 8
  store i32 %ix, i32* %1, align 4
  store i32* %tt, i32** %2, align 8
  %3 = load i32* %1, align 4
  %4 = icmp sge i32 %3, 0
  br i1 %4, label %5, label %14

; <label>:5                                       ; preds = %0
  %6 = load i32* %1, align 4
  %7 = icmp sle i32 %6, 99
  br i1 %7, label %8, label %14

; <label>:8                                       ; preds = %5
  %9 = load i32* %1, align 4
  %10 = sext i32 %9 to i64
  %11 = getelementptr  [100 x i32]* @Array__arr, i32 0, i64 %10
  %12 = load i32* %11
  %13 = load i32** %2, align 8
  store i32 %12, i32* %13
  br label %16

; <label>:14                                      ; preds = %5, %0
  %15 = load i32** %2, align 8
  store i32 0, i32* %15
  br label %16

; <label>:16                                      ; preds = %14, %8
  ret void
}
\end{llvmcode}


\begin{llvmcode}
define void @Array__swap(i32 %ix, i32 %jx) nounwind ssp {
  %1 = alloca i32, align 4
  %2 = alloca i32, align 4
  %temp = alloca i32, align 4
  store i32 %ix, i32* %1, align 4
  store i32 %jx, i32* %2, align 4
  %3 = load i32* %1, align 4
  %4 = icmp sge i32 %3, 0
  br i1 %4, label %5, label %30

; <label>:5                                       ; preds = %0
  %6 = load i32* %1, align 4
  %7 = icmp sle i32 %6, 99
  br i1 %7, label %8, label %30

; <label>:8                                       ; preds = %5
  %9 = load i32* %2, align 4
  %10 = icmp sge i32 %9, 0
  br i1 %10, label %11, label %30

; <label>:11                                      ; preds = %8
  %12 = load i32* %2, align 4
  %13 = icmp sle i32 %12, 99
  br i1 %13, label %14, label %30

; <label>:14                                      ; preds = %11
  %15 = load i32* %2, align 4
  %16 = sext i32 %15 to i64
  %17 = getelementptr  [100 x i32]* @Array__arr, i32 0, i64 %16
  %18 = load i32* %17
  store i32 %18, i32* %temp, align 4
  %19 = load i32* %1, align 4
  %20 = sext i32 %19 to i64
  %21 = getelementptr  [100 x i32]* @Array__arr, i32 0, i64 %20
  %22 = load i32* %21
  %23 = load i32* %2, align 4
  %24 = sext i32 %23 to i64
  %25 = getelementptr  [100 x i32]* @Array__arr, i32 0, i64 %24
  store i32 %22, i32* %25
  %26 = load i32* %temp, align 4
  %27 = load i32* %1, align 4
  %28 = sext i32 %27 to i64
  %29 = getelementptr  [100 x i32]* @Array__arr, i32 0, i64 %28
  store i32 %26, i32* %29
  br label %30

; <label>:30                                      ; preds = %14, %11, %8, %5, %0
  ret void
}

\end{llvmcode}




\subsection{Real example - Bubble Sort}

\begin{pascalcode}
MACHINE
   Bubble
VARIABLES
   vec,sort
INVARIANT
   vec : 0..99 --> 0..99 & sort : 0..1 &
   ( sort = 1 =>  !ii.(ii : 0..98 => vec(ii)<=vec(ii+1)))
INITIALISATION
   vec :: 0..99 --> 0..99 || sort := 0
OPERATIONS
   op_sort =
   ANY sorted_vector WHERE
    sorted_vector : 0..99 --> 0..99 &
    !ii.(ii : 0..98 => sorted_vector(ii)<=sorted_vector(ii+1))
   THEN
    vec := sorted_vector|| sort := 1
   END

END


IMPLEMENTATION
   Bubble_i
REFINES
   Bubble
CONCRETE_VARIABLES
   vec1,sort1
INVARIANT
   vec1 : 0..99 --> 0..99 & sort1 : 0..1 &
   vec = vec1 & sort1 = sort
INITIALISATION
   vec1(0) := 0  ;
   sort1 := 0
OPERATIONS
   op_sort =
   VAR nn, swapped, ii, tmp, vi, vi2 IN
	swapped := 1;
	nn := 100;
	ii:=0;
	WHILE swapped = 1 DO
		swapped := 0;
		ii:= 0;
		nn:= nn-1;

		WHILE ii<= nn DO
		vi :=  vec1(ii);
		vi2 := vec1(ii+1);
		IF vi > vi2 THEN
			tmp:= vec1(ii+1);
			vec1(ii+1):=vec1(ii);
			vec1(ii):=tmp;
			swapped:= 1
		END;
		ii:= ii+1

		INVARIANT 1>0
		VARIANT nn-ii
		END

	INVARIANT 1>0
	VARIANT nn-ii
	END
   END
END
\end{pascalcode}


