\documentclass[10pt,a4paper]{article}
\usepackage[utf8]{inputenc}
\usepackage{amsfonts}
\usepackage{amssymb}

\usepackage[fleqn]{amsmath}
\usepackage{xspace,framed,here,keystroke}
\usepackage{graphicx}
\usepackage[colorlinks=false]{hyperref}
\usepackage{minted}   %Used to syntax highlight



\newcommand{\trad}[2]{\ensuremath{\lVert \textsf{#1} \rVert^{\textit{#2}}}}
\newcommand{\nl}[0]{\ensuremath{\downarrow}}
\newcommand{\mty}[0]{\texttt{""}}
\DeclareMathOperator{\conc}{\diamond}
\DeclareMathOperator{\isdef}{\equiv}
\DeclareMathOperator{\dom}{\mbox{dom}}
\DeclareMathOperator{\lbl}{\mathcal{L}()}
\DeclareMathOperator{\variable}{\mathcal{V}()}
\newcommand{\llvm}[1]{\texttt{#1}}
\newcommand{\B}[1]{\textsf{#1}}
\newcommand{\lalt}[0]{$\langle$\xspace}
\newcommand{\ralt}[0]{$\rangle$\xspace}
\newcommand{\alt}[0]{$\mid\,$}
\newcommand{\ListOf}[1]{$\mbox{#1}^+$}
\newcommand{\nt}[1]{{\normalfont\textit{#1}}}
\newcommand{\Dict}[0]{\mathbb{D}}
\newcommand{\Text}[0]{\mathbb{T}}
\newcommand{\IF}[0]{\textbf{ if }}
\newcommand{\ELSIF}[0]{\textbf{ else if }}
\newcommand{\ELSE}[0]{\textbf{ else }}
\newcommand{\THEN}[0]{\textbf{ then }}
\newcommand{\LET}[0]{\textbf{ let }}
\newcommand{\IN}[0]{\textbf{ in }}
\newcommand{\AND}[0]{\textbf{ and }}
\newcommand{\PH}[1]{\framebox{$#1$}}
\newcommand{\sep}[0]{\otimes}
\newcommand{\intf}[0]{\ensuremath{\mathbb{I}}}
\newcommand{\Global}[0]{\ensuremath{\sf\Gamma}}
\newcommand{\local}[0]{\ensuremath{\sf\lambda}}
\newcommand{\opmap}[0]{\ensuremath{\sf\Omega}}
\newcommand{\developed}[0]{\ensuremath{\textsf{developed}}}
\newcommand{\stateless}[0]{\ensuremath{\textsf{stateless}}}
\newcommand{\importedmodules}[0]{\ensuremath{\textsf{i-mod}}}
\newcommand{\trimportedmodules}[0]{\ensuremath{\textsf{i-mod$\mathsf{\ast}$}}}
\newcommand{\importedinstances}[0]{\ensuremath{\textsf{i-inst}}}
\newcommand{\trimportedinstances}[0]{\ensuremath{\textsf{i-inst$\mathsf{\ast}$}}}
\newcommand{\idx}[0]{\ensuremath{\sf\Pi}}
\newcommand{\state}[0]{\ensuremath{\sf\Theta}}
\newcommand{\stateref}[0]{\ensuremath{\sf\Phi}}
\newcommand{\self}[0]{\ensuremath{\sf\Sigma}}
\newcommand{\init}[0]{\ensuremath{\mathsf{I}}}
\newcommand{\tradi}[2]{\ensuremath{\langle \textsf{#1} \rangle^{\textit{#2}}}}

\newcommand{\DD}[1]{\marginpar{\tiny{DD: #1}}}
\newcommand{\VGM}[1]{\marginpar{\tiny{VGM: #1}}}

\newcommand{\stateindex}[0]{\ensuremath{\sf\tau}}
 
\author{David Déharbe, Valério Medeiros Jr.}
\title{Array}

\begin{document}

%TODO Review the text before present to David

\definecolor{bg}{rgb}{0.95,0.95,0.95}

\usemintedstyle{friendly}

\section{The abstract syntax tree with support for advanced types}

The supported advanced types are array and structures. The nodes Array and 
Structure can be used to declare a variable or to select an element in the array 
or the structure. 


\begin{figure}[H]
  \begin{center}
    {\footnotesize
      \frame{
    \begin{tabular}[t]{ccc}
      \begin{tabular}[t]{rcl}
        \hline
        \multicolumn{3}{|c|}{Machine: \B{Machine}} \\
        \hline
        \B{id} & : & \B{Name} \\
        \B{const} & : & seq \B{Cons} \\
        \B{var} & : & seq \B{Vari} \\
        \B{init} & : & seq \B{Inst} \\
        \B{op} & : & seq \B{Oper} \\
        \B{impl} & : & opt \B{Impl} \\
        \hline
        \multicolumn{3}{|c|}{Implementation: \B{Impl}} \\
        \hline
        \B{id} & : & \B{Name} \\
        \B{impo} & : & seq \B{Import} \\
        \B{const} & : & seq \B{Cons} \\
        \B{var} & : & seq \B{Vari} \\
        \B{init} & : & seq \B{Inst} \\
        \B{op} & : & seq \B{Oper} \\
        \hline
        \multicolumn{3}{|c|}{Imports: \B{Import}} \\
        \hline
        \B{mach} & : & \B{Machine} \\
        \B{pre} & : & opt \B{Name}
        \\
        \hline
        \multicolumn{3}{|c|}{Constant: \B{Cons}} \\
        \hline
        \B{id} & : & \B{Name} \\
        \B{type} & : & \B{Type} \\
        \B{val} & : & \B{Value} \\
        \hline
        \multicolumn{3}{|c|}{Variable: \B{Vari}} \\
        \hline
        \B{id} & : & \B{Name} \\
        \B{type} & : & \B{Type} \\
        \B{scope} & : & \B{Impl} \alt \B{Oper}\\
	\hline
        \multicolumn{3}{|c|}{Operation: \B{Oper}} \\
        \hline
        \B{id} & : & \B{Name} \\
        \B{inp} & : & seq \B{Vari} \\
        \B{out} & : & seq \B{Vari} \\
        \B{body} & : & \B{Inst}
        \\
       \hline
        \multicolumn{3}{|c|}{Instruction: \B{Inst}} \\
        \hline
        \multicolumn{3}{c}{\B{Blk} \alt \B{VarD} \alt \B{If} \alt \B{BEq} \alt } \\
        \multicolumn{3}{c}{\B{Call} \alt \B{While} \alt \B{Case} \alt \B{Skip}} \\
      
       \end{tabular}
      
       \begin{tabular}[t]{rcl}

        \hline
        \multicolumn{3}{|c|}{Block: \B{Blk}} \\
        \hline
        \B{body} & : & seq \B{Inst} \\   
        \hline
        \multicolumn{3}{|c|}{Variable declaration: \B{VarD}} \\
        \hline
        \B{vars} & : & seq \B{Name} \\
        \B{body} & : & seq \B{Inst} \\
	\hline
        \multicolumn{3}{|c|}{If: \B{If}} \\
        \hline
        \B{branches} & : & seq \B{IfBr} \\
        \hline
        \multicolumn{3}{|c|}{Becomes equal: \B{Beq}} \\
        \hline
        \B{lhs} & : & \B{Vari} \\
        \B{rhs} & : & \B{Expr} \\
        \hline
        \multicolumn{3}{|c|}{Operation call: \B{Call}} \\
        \hline
        \B{op} & : & \B{Oper} \\
        \B{inp} & : & seq \B{Expr} \\
        \B{out} & : & seq \B{Name} \\
        \B{inst} & : & opt \B{Impo} \\
        \hline
        \multicolumn{3}{|c|}{Case conditional: \B{Case}} \\
        \hline
        \B{expr} & : & \B{Expr} \\
        \B{branches} & : & seq \B{CaseBr}\\
	\hline
        \multicolumn{3}{|c|}{Case branch: \B{CaseBr}} \\
        \hline
        \B{val} & : & seq \B{Value} \\
        \B{body} & : & \B{Inst}
        \\
        \hline
        \multicolumn{3}{|c|}{While loop: \B{While}} \\
        \hline
        \B{cond} & : & \B{Pred} \\
        \B{body} & : & \B{Inst} \\
        \hline
        \multicolumn{3}{|c|}{If branch: \B{IfBr}} \\
        \hline
        \B{cond} & : & opt \B{Pred} \\
        \B{body} & : & \B{Inst} \\
        \hline
        \multicolumn{3}{|c|}{Expression: \B{Expr}} \\
        \hline
        \multicolumn{3}{c} {\B{Lit} \alt \B{Term} \alt \B{Pred} \alt \B{Vari}} \\
        \hline
        \multicolumn{3}{|c|}{Term expression: \B{Term}} \\
        \hline
        \B{op} & : & \B{ArithOp} \\
        \B{args} & : & seq \B{Exp} \\

      \end{tabular}
       
      \begin{tabular}[t]{rcl}

        \hline
        \multicolumn{3}{|c|}{Predicate: \B{Pred}} \\
        \hline
        \multicolumn{3}{c}{\B{Form} \alt \B{Comp}} \\
        \hline
        \multicolumn{3}{|c|}{Boolean formula: \B{Form}} \\
        \hline
        \B{op} & : & \B{BoolOp} \\
        \B{args} & : & seq \B{Pred} \\
        \hline
        \multicolumn{3}{|c|}{Comparison: \B{Comp}} \\
        \hline
        \B{op} & : & \B{RelOp} \\
        \B{arg1}, \B{arg2} & : & \B{Exp} \\
        \hline
        \multicolumn{3}{|c|}{Data type: \B{Type}} \\
        \hline
        \multicolumn{3}{c}{\B{INT} \alt \B{BOOL}\alt}\\
	\multicolumn{3}{c}{\B{Struct} \alt \B{Array}}\\
	\hline
  	\multicolumn{3}{|c|}{Data structure : \B{Struct}} \\
        \hline
	\B{id} & : &  \B{Name}\\
        \B{elements} & : & seq \B{Name} \\
	\B{type} & : & seq \B{Type}\\	
        \hline
        \multicolumn{3}{|c|}{Data array : \B{Array}} \\
        \hline
	\B{dec} & : &  seq \B{Vector} \\
	\B{type} & : & seq \B{CSet}\\
	\B{indexes} & : & seq \B{STerm}\\
	\hline
  	\multicolumn{3}{|c|}{Vector : \B{Vector}} \\
        \hline
	\B{elements} & : & seq \B{Expr}\\ 
	\hline
	\multicolumn{3}{|c|}{Concrete set : \B{CSet}} \\
	\hline
	\multicolumn{3}{c}{BOOL \alt NAT \alt NAT$_{1}$\alt } \\
	\multicolumn{3}{c}{INT \alt B0\_INTERVAL} \\
	\hline
	\multicolumn{3}{|c|}{B0 interval : \B{B0\_INTERVAL}} \\
	\hline
	\B{start} & : & \B{Exp} \\
	\B{end} & : & \B{Exp} \\
	\hline
	\multicolumn{3}{|c|}{Simple term : \B{STerm}} \\
	\hline
	\B{index} & : &  \B{BOOL} \alt \B{INT}\alt \\ 
	& &  \B{Vari} \alt \B{Cons} \\
      \end{tabular}
    \end{tabular}
  }
  }
    \caption{Abstract syntax structure of a B implementation.  For each abstract
      syntax element, we give the name of the class (e.g. \B{CaseBr} for a
      branch in a case instruction, and the different attributes of the class),
      or a list of the possible sub-classes (e.g. \B{Inst} for the different
      kinds of instructions).
      Sequence and optional attributes are denoted ``seq'' and ``opt'',
      respectively}
    \label{tab:node-attr}
  \end{center}
\end{figure}


\section{The LLVM grammar with support to array multidimensional}


\begin{figure}
  \begin{center}
    \begin{tabular}{rcl}
      \nt{module} & ::= & \ListOf{\nt{item}} \\
      \nt{item} & ::= & \nt{const\_decl} \alt \nt{function\_decl} \alt \nt{type\_def}
      \alt \nt{const\_def} \alt \nt{var\_def} \alt \nt{function\_def} \\
      \nt{const\_decl} & ::= & \nt{name} \llvm{=} \llvm{external} \llvm{constant} \nt{type} \\
      \nt{type\_def} & ::= & \nt{name} \llvm{=} \llvm{type} \nt{type} \\
      \nt{type} & ::= & \llvm{void} \alt \nt{itype} \alt \llvm{\{} \ListOf{\nt{type}} \llvm{\}} \alt \nt{type}\llvm{*} \\
      \nt{const\_def} & ::= & \nt{name} \llvm{=} \llvm{constant} \nt{type} \nt{iliteral} \\
      \nt{var\_def} & ::= & \nt{name} \llvm{=} \llvm{common} \llvm{global} \nt{type} \llvm{zeroinitializer} \\
      \nt{function\_decl} & ::= & \llvm{declare} \nt{type} \nt{name} \llvm{(} \ListOf{\nt{type}} \llvm{)}\\
      \nt{function\_def} & ::= & \llvm{define} \nt{type} \nt{name} \llvm{(} \ListOf{\nt{param}} \llvm{)} \llvm{\{} \ListOf{\nt{block}} \llvm{\}} \\
      \nt{param} & ::= & \nt{type} \nt{name} \\
      \nt{block} & ::= & \nt{lbl} \llvm{:} \ListOf{\nt{inst}} \\
      \nt{inst} & ::=  & \nt{name} \llvm{=} \llvm{alloca} \nt{type} \\
      & \alt & \nt{name} \llvm{=} \lalt \llvm{add} \alt \llvm{sub} \alt \llvm{mul} \alt \llvm{sdiv} \alt \llvm{srem} \ralt \nt{itype} \nt{exp} \llvm{,} \nt{exp} \\
      & \alt & \nt{name} \llvm{=} \llvm{icmp} \lalt \llvm{eq} \alt \llvm{ne} \alt \llvm{sgt} \alt \llvm{sge} \alt \llvm{slt} \alt \llvm{sle} \ralt \llvm{i1} \nt{exp} \llvm{,} \nt{exp}\\
      & \alt & \nt{name} \llvm{=} \llvm{call} \nt{type} \llvm{(} \ListOf{\nt{arg}} \llvm{)} \\
      & \alt & \nt{name} \llvm{=} \llvm{getelementptr} \nt{type} \llvm{*} 
\nt{exp}\llvm{,} \nt{index}\llvm{,} \ListOf{\nt{index}} \\
      & \alt & \nt{name} \llvm{=} \llvm{load} \nt{type} \nt{exp} \\
      & \alt & \llvm{store} \nt{type} \nt{exp}, \nt{type} \llvm{*} \nt{exp} \\
      & \alt & \llvm{br} \llvm{i1} \nt{exp} \llvm{,} \llvm{label} \nt{lbl} \llvm{,} \llvm{label} \nt{lbl} \\
      & \alt & \llvm{br} \llvm{label} \nt{lbl} \\
%       & \alt & \llvm{switch} \nt{type} \nt{exp} \llvm{,} \llvm{branch} \nt{lbl} \llvm{[} \ListOf{\nt{branch}} \llvm{]} \\
      & \alt & \llvm{ret} \lalt \nt{type} \nt{exp} \alt \llvm{void} \ralt \\
      \nt{exp} & ::= & \nt{name} \alt \nt{iliteral} \alt \llvm{getelementptr} 
\llvm{(} \nt{type} \nt{exp} \llvm{,} \nt{index} \llvm{,} \ListOf{\nt{index}} \llvm{)} \\
      \nt{index} & ::= & \nt{itype} \nt{iliteral} \\
      \nt{branch} & ::= & \nt{iliteral} \nt{iliteral} \nt{lbl} \\
      \nt{arg} & ::= & \nt{type} \nt{exp}
    \end{tabular}
  \end{center}
  \caption{Grammar of the target LLVM subset: \nt{itype}, \nt{iliteral}, \nt{lbl}
    and \nt{name} correspond respectively to integer types, integer literals,
    labels and names. Choices are separated by \alt and optionally delimited by
    \lalt and \ralt.  The \ListOf{} superscript denotes a comma-separated list of
    elements of the annotated entity. }
  \label{fig:llvm-grammar}
\end{figure}

\VGM{To support arrays and structures, I 
    
added only + after index.} 

\section{Translation of Array and Structures}


Two approaches are possible to support arrays. An approach is importing
reusable libraries of AtelierB that are reported in the official document of
ClearSy\footnote{\url{http://www.atelierb.eu/ressources/DOC/english/reusable-components-reference.pdf}}.
This approach is supported, but the generated code is more complex and it needs
more intermediate steps to manipulate the data of imported components. Moreover,
this approach allows dynamic allocation for arrays, but it is strongly not
recommended to safety-critical systems. A second approach enjoys the direct 
correspondence between B0 native arrays and LLVM-IR and this approach is presented here.

A concrete array is a total function where the domain indexes the range 
elements. The domain is composed of Cartesian product of sequence of concrete 
sets. The node Data array in AST can represent a declaration of array or a 
selection of element. The first element ``dec'' declares explicitly an array or 
references a variable array. The second element ``type'' defines a sequence of 
array types, where the last element defines the range of function and the other 
elements define the domain. The third element ``indexes'' is optional, and it 
represents the selection of an element in array, or the same type structure 
when it is a structure.

Some examples of array are:
\begin{itemize}
	\item $array1 \in (0..10) \to INT $
	\item $array2 \in (0..10) \times (0..10) \to INT $   (a multidimensional example)
	\item $array3 \in BOOL \times (0..10) \to INT$ (a domain with two index types)
\end{itemize} 

\paragraph{Attributions between array and structure}


The B0 language supports attribution directly between arrays or structure, even though the attribution
must have the same domain type and range type. \VGM{Project Decision: How translate a direct 
attribution between advanced types, for example, arrayA := arrayB ? The Commenc 
uses a function called ``memmove''.  %TODO to add this restriction in become equals.
 } 





\subsection{Relation arrays B0 and LLVM-IR}
\VGM{It must be extended to support structure.}

The arrays in B0 are represented by functions where an index points to an 
element. By default, the element of index not initialized has a zero value. 

This section shows the relation between arrays B0 and LLVM-IR. The sequence of
steps are: defining  an array, getting an element and setting an
element.

\subsubsection{Defining an array}


The essential elements of array are its name, size and type. The
following code declares an array global called ``arr'' with 100 integer
numbers with 32 bits and initialized with zero.

\begin{footnotesize}
\begin{minted}[frame=single,linenos=true,bgcolor=bg]{llvm}
@arr = internal global [100 x i32] zeroinitializer
\end{minted}
\end{footnotesize}
 

Some type examples are:
\begin{itemize}
	\item \verb|[25 x i32]| - Array with 25 elements of 32-bit integer values.
	\item \verb|[4 x [5 x [6 x i16]]] | - Multidimensional array with (4 x 5
x 6) elements of 16-bit integer values.
\end{itemize}


\subsubsection{Getting a element from array}
\VGM{Maybe it can be summarized with text of section 3}

The LLVM-IR instruction \textit{getelementptr} is used to select an element in
the array or structure. A simple example is: 

\begin{footnotesize}
\begin{minted}[frame=single,linenos=true,bgcolor=bg]{llvm}
%4 = getelementptr  ([100 x i32]* @Array__arr, i64 0, i64 %3)
\end{minted}
\end{footnotesize}


The first parameter defines the type and name of the variable, and the others are
indexes to select the element in the data type. The second parameter (i64 0)
selects the first element in the pointer (* @Array\_\_arr). The last element,
(i64 \%3) selects the value in position obtained from \%3 of the array of 32-bit
integer values. When the array is multidimensional, the news index must be added
separated by comma to get the specific element.

\iffalse
\begin{verbatim}
<temporary variable> = getelementptr  
  [<number of elements> x <element type>]   * @<variable_name>, 
  <type and index of pointer> <index_value>,
  <type and index. ex:i64>  <index_value>
\end{verbatim}
\fi

\subsubsection{Setting an element to array}

The following blocks illustrate a B code setting and getting values with array
``arr''  and its translation to LLVM-IR. 
\begin{footnotesize}

\begin{minted}[frame=single,bgcolor=bg,linenos=true]{pascal}
  arr(2):=5;
  i = arr(2);
\end{minted}
\end{footnotesize}

\begin{footnotesize}
\begin{minted}[frame=single,bgcolor=bg,linenos=true]{llvm}
  store i32 5, i32* getelementptr  ([100 x i32]* @arr, i32 0, i64 2)
  %1 = load i32* getelementptr  ([100 x i32]* @arr, i32 0, i64 2)
  store i32 %1, i32* @i, align 4
\end{minted}
\end{footnotesize}



\subsection{Draft of Rules}

\paragraph{An atomic expression}

\begin{align*}
\lefteqn{\trad{n}{\B{Name}} \isdef \LET t = \trad{n.type}{\B{Type}} \IN} \\
& \IF \B{n} \mbox{ is a constant } \THEN \trad{n.value}{\B{Expr}} \\
& \ELSIF \B{n} \mbox{ is a local variable } \THEN
\llvm{""} \sep \local(\B{n}) \sep t \\
& \ELSE (\B{n} \mbox{ is a primitive state variable }) \LET p = \variable \AND v = \variable \IN \\
& \quad \llvm{"\PH{p} = getelementptr  \PH{\stateref(\B{n.root})} \%self\$, i32 0, i32 \PH{\idx(\B{n})} \nl} \\
& \quad \llvm{\PH{v} = load \PH{t}* \PH{p}\nl"} \sep v \sep t\\
& \ELSE (\B{n} \mbox{ is an array or structure state variable }) \LET p = \variable \AND v = \variable \IN \\
& \quad \llvm{"\PH{p} = getelementptr \PH{\state(\B{n.root})} \%self\$,
\PH{\stateindex(n)} \nl} \\
& \quad \llvm{\PH{v} = load \PH{t}* \PH{p}\nl"} \sep v \sep t
\
\end{align*}

Definition of a new function:

\begin{itemize}
\item $\stateindex(\B{n}) \isdef$ to get a sequence of index to select the 
element. If $\B{n} $ is integer vector ``arr'' with 100 elements selecting the 
element 1 then the result is this text ``, i32 0, i64 1''. It means to select 
the element in dimension zero and position one of array, where the index types are 
respectively ``i32'' and ``i64''. More detailed examples are presented in the reference 
manual of LLVM-IR\ref{}.

\end{itemize}

\subsubsection{Issues and problems}
\begin{itemize}
	\item To develop support to structures (more easy)
	\item To integrate it in translation.tex
	\item To develop support to copy all  of an advanced types
	\item To identify the rules that must be updated:
	

\end{itemize}


\section*{Attachments}




%\iffalse  % This content is not visible to reader.
A generic example available at repository ProB\footnote{
\url{https://github.com/bendisposto/probparsers/blob/develop/bparser/src/test/resources/parsable/Array.mch}}. 
This example can be used in approach using imported components.

\subsection{A very simple and generic example}


\begin{minted}[frame=single,bgcolor=bg,linenos=true]{pascal}
MACHINE
   Array(maxindex, TYPE)
CONSTRAINTS maxindex : NAT
VARIABLES arr
INVARIANT arr : 1..maxindex +-> TYPE
INITIALISATION arr := {}
OPERATIONS 

set(ix,tt)=
        PRE ix : 1..maxindex & tt : TYPE
        THEN arr := arr <+ {ix |-> tt}
        END ;

tt <-- read(ix) =
        PRE ix : dom (arr)
        THEN tt := arr(ix)
        END;

swap(ix, jx)=
        PRE ix : dom(arr) & jx : dom(arr)
        THEN arr := arr <+ {ix |-> arr(jx), jx|-> arr(ix)}
        END

END
\end{minted}
%\fi
\subsection{A  simple example and its LLVM-IR}
\footnotesize{
\begin{minted}[frame=single,bgcolor=bg,linenos=true]{pascal}
MACHINE
   Array
VARIABLES arr

INVARIANT 
    arr : ((0..99) --> 0..1000)

INITIALISATION arr := {1|->100}

OPERATIONS 

set(ix,tt)=
        PRE ix : (0..99) & tt : (0..1000)
        THEN arr := arr <+ {ix |-> tt}
        END ;

tt <-- read(ix) =
        PRE ix : (0..99)
        THEN tt := arr(ix)
        END;

swap(ix, jx)=
        PRE ix : (0..99) & jx : (0..99)
        THEN arr := arr <+ {ix |-> arr(jx), jx|-> arr(ix)}
        END

END
\end{minted}

\begin{minted}[frame=single,bgcolor=bg,linenos=true]{pascal}
IMPLEMENTATION 
   Array_i
REFINES
   Array
CONCRETE_VARIABLES
   arr
INVARIANT 
    arr :  ((0..99) --> 0..1000)
INITIALISATION
   arr(1) := 100
OPERATIONS
   set ( ix , tt ) =
   IF
       ix >= 0 & ix <=99 & tt >= 0  & tt <= 1000
   THEN
       
      arr(ix) := tt 
   END;

   tt <-- read ( ix ) =
   IF
       ix >= 0 & ix <=99
   THEN
       tt := arr ( ix )
   ELSE
       tt := 0
   END;

   swap ( ix , jx ) =
   IF
        ix >= 0 & ix <=99 & jx >= 0 & jx <=99 
   THEN

       VAR temp IN
           temp := arr(jx);
           arr(jx) := arr(ix);
           arr(ix) := temp
       END
   END


END


\end{minted}




\begin{minted}[frame=single,bgcolor=bg,linenos=true]{llvm}

@Array__arr = internal global [100 x i32] zeroinitializer, align 16

define void @Array__INITIALISATION() nounwind ssp {
  store i32 100, i32* getelementptr  ([100 x i32]* @Array__arr, i32 0, i64 1)
  ret void
}

define void @Array__set(i32 %ix, i32 %tt) nounwind ssp {
  %1 = alloca i32, align 4
  %2 = alloca i32, align 4
  store i32 %ix, i32* %1, align 4
  store i32 %tt, i32* %2, align 4
  %3 = load i32* %1, align 4
  %4 = icmp sge i32 %3, 0
  br i1 %4, label %5, label %19

; <label>:5                                       ; preds = %0
  %6 = load i32* %1, align 4
  %7 = icmp sle i32 %6, 99
  br i1 %7, label %8, label %19

; <label>:8                                       ; preds = %5
  %9 = load i32* %2, align 4
  %10 = icmp sge i32 %9, 0
  br i1 %10, label %11, label %19

; <label>:11                                      ; preds = %8
  %12 = load i32* %2, align 4
  %13 = icmp sle i32 %12, 1000
  br i1 %13, label %14, label %19

; <label>:14                                      ; preds = %11
  %15 = load i32* %2, align 4
  %16 = load i32* %1, align 4
  %17 = sext i32 %16 to i64
  %18 = getelementptr  [100 x i32]* @Array__arr, i32 0, i64 %17
  store i32 %15, i32* %18
  br label %19
; <label>:19                                      ; preds = %14, %11, %8, %5, %0
  ret void
}
\end{minted}


\begin{minted}[frame=single,bgcolor=bg,linenos=true]{llvm}

define void @Array__read(i32 %ix, i32* %tt) nounwind ssp {
  %1 = alloca i32, align 4
  %2 = alloca i32*, align 8
  store i32 %ix, i32* %1, align 4
  store i32* %tt, i32** %2, align 8
  %3 = load i32* %1, align 4
  %4 = icmp sge i32 %3, 0
  br i1 %4, label %5, label %14

; <label>:5                                       ; preds = %0
  %6 = load i32* %1, align 4
  %7 = icmp sle i32 %6, 99
  br i1 %7, label %8, label %14

; <label>:8                                       ; preds = %5
  %9 = load i32* %1, align 4
  %10 = sext i32 %9 to i64
  %11 = getelementptr  [100 x i32]* @Array__arr, i32 0, i64 %10
  %12 = load i32* %11
  %13 = load i32** %2, align 8
  store i32 %12, i32* %13
  br label %16

; <label>:14                                      ; preds = %5, %0
  %15 = load i32** %2, align 8
  store i32 0, i32* %15
  br label %16

; <label>:16                                      ; preds = %14, %8
  ret void
}
\end{minted}


\begin{minted}[frame=single,bgcolor=bg,linenos=true]{llvm}

define void @Array__swap(i32 %ix, i32 %jx) nounwind ssp {
  %1 = alloca i32, align 4
  %2 = alloca i32, align 4
  %temp = alloca i32, align 4
  store i32 %ix, i32* %1, align 4
  store i32 %jx, i32* %2, align 4
  %3 = load i32* %1, align 4
  %4 = icmp sge i32 %3, 0
  br i1 %4, label %5, label %30

; <label>:5                                       ; preds = %0
  %6 = load i32* %1, align 4
  %7 = icmp sle i32 %6, 99
  br i1 %7, label %8, label %30

; <label>:8                                       ; preds = %5
  %9 = load i32* %2, align 4
  %10 = icmp sge i32 %9, 0
  br i1 %10, label %11, label %30

; <label>:11                                      ; preds = %8
  %12 = load i32* %2, align 4
  %13 = icmp sle i32 %12, 99
  br i1 %13, label %14, label %30

; <label>:14                                      ; preds = %11
  %15 = load i32* %2, align 4
  %16 = sext i32 %15 to i64
  %17 = getelementptr  [100 x i32]* @Array__arr, i32 0, i64 %16
  %18 = load i32* %17
  store i32 %18, i32* %temp, align 4
  %19 = load i32* %1, align 4
  %20 = sext i32 %19 to i64
  %21 = getelementptr  [100 x i32]* @Array__arr, i32 0, i64 %20
  %22 = load i32* %21
  %23 = load i32* %2, align 4
  %24 = sext i32 %23 to i64
  %25 = getelementptr  [100 x i32]* @Array__arr, i32 0, i64 %24
  store i32 %22, i32* %25
  %26 = load i32* %temp, align 4
  %27 = load i32* %1, align 4
  %28 = sext i32 %27 to i64
  %29 = getelementptr  [100 x i32]* @Array__arr, i32 0, i64 %28
  store i32 %26, i32* %29
  br label %30

; <label>:30                                      ; preds = %14, %11, %8, %5, %0
  ret void
}

\end{minted}

}


\subsection{Real example - Bubble Sort}

\begin{footnotesize}
\begin{minted}[frame=single,bgcolor=bg,linenos=true]{pascal}
MACHINE
   Bubble
VARIABLES
   vec,sort
INVARIANT
   vec : 0..99 --> 0..99 & sort : 0..1 &
   ( sort = 1 =>  !ii.(ii : 0..98 => vec(ii)<=vec(ii+1)))
INITIALISATION
   vec :: 0..99 --> 0..99 || sort := 0
OPERATIONS
   op_sort =
   ANY sorted_vector WHERE
    sorted_vector : 0..99 --> 0..99 &
    !ii.(ii : 0..98 => sorted_vector(ii)<=sorted_vector(ii+1))
   THEN
    vec := sorted_vector|| sort := 1
   END

END


IMPLEMENTATION
   Bubble_i
REFINES
   Bubble
CONCRETE_VARIABLES
   vec1,sort1
INVARIANT
   vec1 : 0..99 --> 0..99 & sort1 : 0..1 &
   vec = vec1 & sort1 = sort
INITIALISATION
   vec1(0) := 0  ;
   sort1 := 0
OPERATIONS
   op_sort =
   VAR nn, swapped, ii, tmp, vi, vi2 IN
	swapped := 1;
	nn := 100;
	ii:=0;
	WHILE swapped = 1 DO
		swapped := 0;
		ii:= 0;
		nn:= nn-1;

		WHILE ii<= nn DO
		vi :=  vec1(ii);
		vi2 := vec1(ii+1);
		IF vi > vi2 THEN
			tmp:= vec1(ii+1);
			vec1(ii+1):=vec1(ii);
			vec1(ii):=tmp;
			swapped:= 1
		END;
		ii:= ii+1

		INVARIANT 1>0
		VARIANT nn-ii
		END

	INVARIANT 1>0
	VARIANT nn-ii
	END
   END
END


\end{minted}
\end{footnotesize}


\end{document}
