\documentclass{llncs}

\usepackage[utf8]{inputenc}
\usepackage[english]{babel}
\usepackage{amssymb}
\usepackage[fleqn]{amsmath}
\usepackage[colorlinks=false]{hyperref}
\usepackage{xspace,framed,here,keystroke}
\pagestyle{headings}
\title{Translation of B Implementations to the LLVM: Control and Modularity}
\author{David Déharbe \and Valério Medeiros Jr} 
\institute{Program of Graduate Studies in Systems and Computing (PPgSC) \\
  Federal University of Rio Grande do Norte \\
  Natal (Brazil)}
\date{May 2013}

\newcommand{\trad}[2]{\ensuremath{\lVert \textsf{#1} \rVert^{\textit{#2}}}}
\newcommand{\nl}[0]{\text{\Return}}
\newcommand{\mty}[0]{\texttt{""}}
\DeclareMathOperator{\conc}{\diamond}
\DeclareMathOperator{\isdef}{\equiv}
\DeclareMathOperator{\dom}{\mbox{dom}}
\DeclareMathOperator{\lbl}{\mathcal{L}()}
\DeclareMathOperator{\variable}{\mathcal{V}()}
\newcommand{\llvm}[1]{\texttt{#1}}
\newcommand{\B}[1]{\textsf{#1}}
\newcommand{\lalt}[0]{$\langle$\xspace}
\newcommand{\ralt}[0]{$\rangle$\xspace}
\newcommand{\alt}[0]{$\mid\,$}
\newcommand{\ListOf}[1]{$\mbox{#1}^+$}
\newcommand{\nt}[1]{{\normalfont\textit{#1}}}
\newcommand{\Dict}[0]{\mathbb{D}}
\newcommand{\Text}[0]{\mathbb{T}}
\newcommand{\IF}[0]{\textbf{ if }}
\newcommand{\ELSIF}[0]{\textbf{ else if }}
\newcommand{\ELSE}[0]{\textbf{ else }}
\newcommand{\THEN}[0]{\textbf{ then }}
\newcommand{\LET}[0]{\textbf{ let }}
\newcommand{\IN}[0]{\textbf{ in }}
\newcommand{\AND}[0]{\textbf{ and }}
\newcommand{\PH}[1]{\framebox{$#1$}}
\newcommand{\sep}[0]{\otimes}
\newcommand{\intf}[0]{\ensuremath{\mathbb{I}}}
\newcommand{\Global}[0]{\ensuremath{\sf\Gamma}}
\newcommand{\local}[0]{\ensuremath{\sf\lambda}}
\newcommand{\opmap}[0]{\ensuremath{\sf\Omega}}
\newcommand{\idx}[0]{\ensuremath{\sf\Pi}}
\newcommand{\state}[0]{\ensuremath{\sf\Sigma}}
\newcommand{\tradi}[2]{\ensuremath{\langle \textsf{#1} \rangle^{\textit{#2}}}}

\newcommand{\DD}[1]{\marginpar{\tiny{DD: #1}}}

\begin{document}
\maketitle

\begin{abstract}
  The aim of this work is to lay the foundations of a multi-platform code
  generator for the B method. In particular, this paper presents a translation
  procedure from a large subset of the B language for implementations towards
  LLVM source code. This translation is defined formally as a set of rules
  defined recursively on the abstract syntax for B implementations. It handles
  the following elements of the B language: simple data types, imperative
  instructions and component compositions. 
\end{abstract}

\section{Introduction}

B is a formal refinement-based software design method~\cite{Abrial1996}. It has
a single language encompassing abstract constructs suitable for specification,
as well as classic imperative constructs for computer programming. The starting
point of a B development is a specification, called a \emph{machine}, that is
incrementally refined to an \emph{implementation}, where only imperative-like
constructs may be employed~\cite{Clearsy}. Such implementation may then be
translated to source code in a programming language, say C~\cite{ComenC} or
Ada. All steps in the B method are formally verified using certified theorem
proving technologies. However the translation to some programming language and
its subsequent compilation to the target platform do not benefit from the same
mathematical rigor. In practice, redundancy in the tool chains and execution
platform are employed to increase the level of confidence to the desired levels.

The goal of this work is to connect the B method to the LLVM compilation
framework~\cite{Lattner04LLVM}.  LLVM is a free, open-source project upon which
many compiling contemporary technologies and tools are based. LLVM provides an
intermediate assembly language upon which may be applied techniques such as
optimization, static analysis, code generation, debugging, etc. Providing a
connection from the B method to this framework gives the users of the B method
access to these technologies. Our approach is to define a translation from B0
(the subset of the B language that is used to describe imperative programs) to
the LLVM intermediate format.

Considering the correctness of the proposed translation, we shall rely upon
human inspection for the time being. The standard uniform semantic framework for
B and LLVM that would be necessary to carry out a proof of this correctness does
not exist (yet). A possible approach to mitigate the risk of having an error in
the translation would be to include assertions in the target code. Indeed the
artifacts produced in the B development have many such assertions and it would
be possible to translate at least some of them (B has an assertion instruction
that introduces additional verification conditions in the development). The
produced code could then be (at least partially) verified by testing the absence
of assertion violations.

\paragraph{Overview.} Following this introduction, the paper is organized as
follows. The implementation constructs of the B language, and selected elements
of the LLVM are presented in section~\ref{sec:b-impl}
and~\ref{sec:llvm}. Section~\ref{sec:overview} presents a bird's-eye view that
is then detailed in sections~\ref{sec:module} (for component modularity)
~\ref{sec:data} (for data aspects),~\ref{sec:expr} (expressions) and
instructions~\ref{sec:control}. We conclude and discuss future work
in~\ref{sec:conclusion}.

\section{Implementation Language in the B-method}
\label{sec:b-impl}

This section describes the B implementation constructs that are considered in
the compilation to the LLVM. This description introduces the notations used in
the definition of the translation. The elements of the B language are denoted
using a \B{sans serif} font. 

Figure~\ref{tab:node-attr} summarizes the structure
of the abstract syntax that is relevant in the definition of the proposed
translation and introduces identifiers used in the definition of the
translation. For instance, and for the purpose of the translation, the relevant
components of a B implementation are: \B{id}, the name identifying the
implementation within the project; \B{see}, a set of references to other modules
containing definitions used in the implementation; \B{const}, a set of constant
names, their type and their values; \B{imp}, a sequence of optionally
prefixed B machines corresponding to external modules instantiated in the
implementation; \B{var}, a set of variable names, their type and possibly
additional functional restrictions forming the rest of the state; \B{init}, a
sequence of instructions executed upon initialization of the implementation;
\B{op}, a set of operations, each being an algorithmic description of the
different functionalities provided by the implementation.

The initialization and the operations are defined as (sequences of)
instructions. Operations also have a name, inputs and outputs. Possible
instructions are the classic imperative constructs: variable assignment, if and
case conditional, while loop, block with or without local variables, and
operation calls. The if constructs may have several branches, each with a
condition as well as an optional else branch. Called operations may be either
local operations or operations of instances of external modules, in this case
they have an optional prefix identifying the module. 

Note that the translation presented in this paper only considers types \B{INT}
(integer) and \B{BOOL} (Boolean). Accordingly, the expression language consists
in arithmetic operations \B{-} (unary and binary), \B{+}, \B{*}, \B{/}, \B{mod},
% \B{**} (exponentiation), 
\B{succ} and \B{pred} and Boolean operations
\B{$\land$}, \B{$\lor$}, \B{$\neg$} as well as relations \B{=}, \B{$\neq$},
\B{$>$}, \B{$<$}, \B{$\leq$} and \B{$\neq$}. Atomic expressions are identifiers,
Boolean constants \B{FALSE} and \B{TRUE}, integer literals, and integer
constants \B{MAXINT} and \B{MININT}. The development through the application of
the B method ensures the absence of overflows and underflows.

\begin{figure}[H]
  \begin{center}
    {\footnotesize
      \frame{
    \begin{tabular}[t]{ccccc}
      \begin{tabular}[t]{rcl}
        \hline
        \multicolumn{3}{|c|}{Implementation: \B{Impl}} \\
        \hline
        \B{id} & : & \B{Name} \\
        \B{see} & : & seq \B{Name} \\
        \B{imp} & : & seq \B{Impo} \\
        \B{const} & : & seq \B{Cons} \\
        \B{var} & : & seq \B{Vari} \\
        \B{init} & : & seq \B{Inst} \\
        \B{op} & : & seq \B{Oper} \\
        \hline
        \multicolumn{3}{|c|}{Imported machine: \B{Impo}} \\
        \hline
        \B{id} & : & \B{Name} \\
        \B{pre} & : & opt \B{Name} \\
        \hline
        \multicolumn{3}{|c|}{Constant: \B{Cons}} \\
        \hline
        \B{id} & : & \B{Name} \\
        \B{type} & : & \B{Type} \\
        \B{val} & : & \B{Value} \\
        \hline
        \multicolumn{3}{|c|}{Variable: \B{Vari}} \\
        \hline
        \B{id} & : & \B{Name} \\
        \B{type} & : & \B{Type} \\
        \B{scope} & : & \B{Impl} \alt \B{Oper}
        \\
      \end{tabular}
      & &
      \begin{tabular}[t]{rcl}
        \hline
        \multicolumn{3}{|c|}{Operation: \B{Oper}} \\
        \hline
        \B{id} & : & \B{Name} \\
        \B{inp} & : & seq \B{Vari} \\
        \B{out} & : & seq \B{Vari} \\
        \B{body} & : & \B{Inst}
        \\
        \hline
        \multicolumn{3}{|c|}{Instruction: \B{Inst}} \\
        \hline
        \multicolumn{3}{c}{\B{Blk} \alt \B{VarD} \alt \B{If} \alt \B{BEq} \alt } \\
        \multicolumn{3}{c}{\B{Call} \alt \B{While} \alt \B{Case}} \\
        \hline
        \multicolumn{3}{|c|}{Block: \B{Blk}} \\
        \hline
        \B{body} & : & seq \B{Inst} \\
        \hline
        \multicolumn{3}{|c|}{Variable declaration: \B{VarD}} \\
        \hline
        \B{vars} & : & seq \B{Name} \\
        \B{body} & : & seq \B{Inst} \\
        \hline
        \multicolumn{3}{|c|}{If: \B{If}} \\
        \hline
        \B{branches} & : & seq \B{IfBr} \\
        \hline
        \multicolumn{3}{|c|}{Becomes equal: \B{Beq}} \\
        \hline
        \B{lhs} & : & \B{Vari} \\
        \B{rhs} & : & \B{Expr} \\
        \hline
        \multicolumn{3}{|c|}{Operation call: \B{Call}} \\
        \hline
        \B{op} & : & \B{Name} \\
        \B{pre} & : & opt \B{Name} \\
        \B{inp} & : & seq \B{Expr} \\
        \B{out} & : & seq \B{Name}
        \\
%         \\
%         \hline
%         \multicolumn{3}{|c|}{Case conditional: \B{Case}} \\
%         \hline
%         \B{expr} & : & \B{Expr} \\
%         \B{branches} & : & seq \B{CaseBr}
      \end{tabular}
      & &
      \begin{tabular}[t]{rcl}
        \hline
        \multicolumn{3}{|c|}{While loop: \B{While}} \\
        \hline
        \B{cond} & : & \B{Pred} \\
        \B{body} & : & \B{Inst}
        \\
        \hline
        \multicolumn{3}{|c|}{If branch: \B{IfBr}} \\
        \hline
        \B{cond} & : & opt \B{Pred} \\
        \B{body} & : & \B{Inst} \\
%         \hline
%         \multicolumn{3}{|c|}{Case branch: \B{CaseBr}} \\
%         \hline
%         \B{val} & : & seq \B{Value} \\
%         \B{body} & : & \B{Inst} \\
        \hline
        \multicolumn{3}{|c|}{Expression: \B{Expr}} \\
        \hline
        \multicolumn{3}{c} {\B{Lit} \alt \B{Term} \alt \B{Pred} \alt \B{Vari}} \\
        \hline
        \multicolumn{3}{|c|}{Term expression: \B{Term}} \\
        \hline
        \B{op} & : & \B{ArithOp} \\
        \B{args} & : & seq \B{Exp} \\
        \hline
        \multicolumn{3}{|c|}{Predicate: \B{Pred}} \\
        \hline
        \multicolumn{3}{c}{\B{Form} \alt \B{Comp}} \\
        \hline
        \multicolumn{3}{|c|}{Boolean formula: \B{Form}} \\
        \hline
        \B{op} & : & \B{BoolOp} \\
        \B{args} & : & seq \B{Pred} \\
        \hline
        \multicolumn{3}{|c|}{Comparison: \B{Comp}} \\
        \hline
        \B{op} & : & \B{RelOp} \\
        \B{arg1}, \B{arg2} & : & \B{Exp} \\
        \hline
        \multicolumn{3}{|c|}{Data type: \B{Type}} \\
        \hline
        \multicolumn{3}{c}{\B{INT} \alt \B{BOOL}}
      \end{tabular}
    \end{tabular}
  }
  }
    \caption{Abstract syntax structure of a B implementation.  For each abstract
      syntax element, we give the name of the class (e.g. \B{CaseBr} for a
      branch in a case instruction, and the different attributes of the class),
      or a list of the possible sub-classes (e.g. \B{Inst} for the different
      kinds of instructions).
      Sequence and optional attributes are denoted ``seq'' and ``opt'',
      respectively}
    \label{tab:node-attr}
  \end{center}
\end{figure}
\DD{Restore \B{Case} and \B{CaseBr}.}

\section{Target LLVM Subset}
\label{sec:llvm}

LLVM aims to be a general, language agnostic, intermediate representation for
compilers. The compiler front-end translates the source programming language to
LLVM source code, upon which optimization and other static analysis tasks may be
performed, and the compiler back-end translates to the target platform assembly
language. One key feature of LLVM is that it is a single-static assignment (SSA)
language: a variable may only be assigned once. So for each assignment to a
variable found in the source language, the front-end shall generate a new
temporary variable in the LLVM program. For illustration,
figure~\ref{fig:ex-llvm} presents a LLVM program (for clarity, the corresponding
C code appears on the left) with two temporary variables \llvm{\%0} and
\llvm{\%1}.
\begin{figure}
\vspace*{-3em}
  \begin{minipage}[t]{.3\textwidth}
    \begin{tabbing}
      foo \= \kill  \\
      void inc(int * pi) \{\\
      \> *pi += 1; \\
      \}\end{tabbing}
  \end{minipage}
  \hfill
  \begin{minipage}[t]{.6\textwidth}
    \begin{tabbing}
      foo \= \kill  \\
      \llvm{define void @inc(i32* \%pi) nounwind \{} \\
      \llvm{entry:}                      \\
      \> \llvm{\%0 = load i32* \%pi}     \\
      \> \llvm{\%1 = add i32 \%0, 1}     \\
      \> \llvm{store i32 \%1, i32* \%pi} \\
      \> \llvm{ret void} \\
      \llvm{\}}
    \end{tabbing}
  \end{minipage}
  \caption{Simple example of a C function and its corresponding LLVM function.
    The first line contains the signature: return type \llvm{void}, the name
    \llvm{@inc} and one parameter named \llvm{\%pi} and typed \llvm{i32*}.  Next
    is the body with a single block, labeled \llvm{entry}, and temporary
    variables \llvm{\%0} and \llvm{\%1}, created in the conversion to SSA. The
    block has four instructions: \llvm{load}, \llvm{add}, \llvm{store} and
    \llvm{ret}. For instance, \llvm{\%1 = add i32 \%0, 1} is an addition
    (\llvm{add}), has result type \llvm{i32} and assigns to
    \llvm{\%1} the sum of variable \llvm{\%0} and integer
    literal \llvm{1}.}
  \label{fig:ex-llvm}
\end{figure}

The introduction to the LLVM that follows is focused in the features
that are relevant for the representation of B
implementations. Figure~\ref{fig:llvm-grammar} presents a grammar of this subset
of the LLVM. 

\begin{figure}
  \begin{center}
    \begin{tabular}{rcl}
      \nt{module} & ::= & \ListOf{\nt{item}} \\
      \nt{item} & ::= & \nt{const\_decl} \alt \nt{function\_decl} \alt \nt{type\_def} 
      \alt \nt{const\_def} \alt \nt{var\_def} \alt \nt{function\_def} \\
      \nt{const\_decl} & ::= & \nt{name} \llvm{=} \llvm{external} \llvm{constant} \nt{type} \\
      \nt{type\_def} & ::= & \nt{name} \llvm{=} \llvm{type} \nt{type} \\
      \nt{type} & ::= & \llvm{void} \alt \nt{itype} \alt \llvm{\{} \ListOf{\nt{type}} \llvm{\}} \alt \nt{type}\llvm{*} \\
      \nt{const\_def} & ::= & \nt{name} \llvm{=} \llvm{constant} \nt{type} \nt{iliteral} \\
      \nt{var\_def} & ::= & \nt{name} \llvm{=} \llvm{common} \llvm{global} \nt{type} \llvm{zeroinitializer} \\
      \nt{function\_decl} & ::= & \llvm{declare} \nt{type} \nt{name} \llvm{(} \ListOf{\nt{type}} \llvm{)}\\
      \nt{function\_def} & ::= & \llvm{define} \nt{type} \nt{name} \llvm{(} \ListOf{\nt{param}} \llvm{)} \llvm{\{} \ListOf{\nt{block}} \llvm{\}} \\
      \nt{param} & ::= & \nt{type} \nt{name} \\
      \nt{block} & ::= & \nt{lbl} \llvm{:} \ListOf{\nt{inst}} \\
      \nt{inst} & ::=  & \nt{name} \llvm{=} \llvm{alloca} \nt{type} \\
      & \alt & \nt{name} \llvm{=} \lalt \llvm{add} \alt \llvm{sub} \alt \llvm{mul} \alt \llvm{sdiv} \alt \llvm{srem} \ralt \nt{itype} \nt{exp} \llvm{,} \nt{exp} \\
      & \alt & \nt{name} \llvm{=} \llvm{icmp} \lalt \llvm{eq} \alt \llvm{sub} \alt \llvm{mul} \alt \llvm{sdiv} \alt \llvm{srem} \ralt \llvm{i1} \nt{exp} \llvm{,} \nt{exp}\\
      & \alt & \nt{name} \llvm{=} \llvm{call} \nt{type} \llvm{(} \ListOf{\nt{arg}} \llvm{)} \\
      & \alt & \nt{name} \llvm{=} \llvm{getelementptr} \nt{type} \llvm{*} \nt{exp}\llvm{,} \nt{index}\llvm{,} \nt{index} \\
      & \alt & \nt{name} \llvm{=} \llvm{load} \nt{type} \nt{exp} \\
      & \alt & \llvm{store} \nt{type} \nt{exp}, \nt{type} \llvm{*} \nt{exp} \\
      & \alt & \llvm{br} \llvm{i1} \nt{exp} \llvm{,} \llvm{label} \nt{lbl} \llvm{,} \llvm{label} \nt{lbl} \\
      & \alt & \llvm{br} \llvm{label} \nt{lbl} \\
%       & \alt & \llvm{switch} \nt{type} \nt{exp} \llvm{,} \llvm{branch} \nt{lbl} \llvm{[} \ListOf{\nt{branch}} \llvm{]} \\
      & \alt & \llvm{ret} \lalt \nt{type} \nt{exp} \alt \llvm{void} \ralt \\
      \nt{exp} & ::= & \nt{name} \alt \nt{iliteral} \alt \llvm{getelementptr} \llvm{(} \nt{type} \nt{exp} \llvm{,} \nt{index} \llvm{,} \nt{index} \llvm{)} \\
      \nt{index} & ::= & \nt{itype} \nt{iliteral} \\
      \nt{branch} & ::= & \nt{iliteral} \nt{iliteral} \nt{lbl} \\
      \nt{arg} & ::= & \nt{type} \nt{exp}
    \end{tabular}
  \end{center}
  \caption{Grammar of the target LLVM subset: \nt{itype}, \nt{iliteral}, \nt{lbl}
    and \nt{name} correspond respectively to integer types, integer literals,
    labels and names. Choices are separated by \alt and optionally delimited by
    \lalt and \ralt.  The \ListOf{} superscript denotes a comma-separated list of
    elements of the annotated entity.}
  \label{fig:llvm-grammar}
\end{figure}

LLVM programs are organized into modules, one per translation unit. A module may
contain declarations of external entities (functions, constants) and definitions
of internal items (functions, variables, constants). All data is explicitly
typed. The name and type of external entities must be declared.  All names,
e.g. non-reserved identifiers, shall start with \llvm{@}, when they are global,
or\llvm{\%}, when they are local. For instance, \llvm{@max = external constant
  i32} declares \llvm{@max} as a 32-bit integer constant and \llvm{declare void
  @inc(i32*)} declares \llvm{@inc} as a function having as parameter a pointer
to an integer and \llvm{void} as a return type.

The type system contains the empty type \llvm{void}, an infinite, countable
number of integer types, one for each possible bit width (e.g. \llvm{i8} is the
type for 8-bit integers), and type constructors pointer (monadic \llvm{$\cdot$
  *}) and structure (polyadic \llvm{\{$\cdots$\}}). For instance \llvm{\{ i8*,
  i8, i8 \}} is the type for structures with three fields, the first having as
type pointer to \llvm{i8}. Grammar rule \nt{type\_def} states how types are
named, e.g.: \llvm{\%T1 = type \{i32, i32\}} and \llvm{\%T2 = type \{\%T1 *,
  \%T1 *\}}. Note that there is no \llvm{void *} type in the LLVM, instead
pointer values are of integer types. 

Local entities are constants, variables or functions. An example of constant
definition is \llvm{@secret = constant i32 42}, it is composed of a name, type
and value. The elements of a variable definition are the name, type and code
generation attributes, e.g \llvm{@count = common global i32 zeroinitializer}.
Attributes \llvm{common}, \llvm{global} and \llvm{zeroinitializer} provide
information for target code generation: linkage type (the variable is merged
with other of the same name), the scope (global) and initialization of the
variable (zeroed is mandatory for common linkage). LLVM defines other attributes
for these aspects, but they are not needed in this work and their presentation
is omitted.  Function definitions are composed of the signature and body. The
signature contains the return type, name, parameters, and attributes for target
code generation. Attribute \llvm{nounwind} states that the function never
generate exceptions. The body is a sequence of blocks of instructions in
single-static assignment (SSA) form, i.e. are such that each variable may be
assigned exactly once.

The grammar rule \nt{inst} describes the different kind of
instructions. Instruction \llvm{alloca} allocates a memory block the size of the
given type from the stack segment. This and the following classes of
instructions assign the resulting value to a fresh variable. Arithmetic
operations are binary, and include signed quotient and reminder. Arithmetic
comparisons compare two values according to the first operand and return a 1-bit
integer value.  Instruction \llvm{call} invokes the given function with the
given arguments and assigns the result to a fresh variable, which is of the
given type.  In general, \llvm{getelementptr} gets the address of an element in
an aggregate object through indexing. In the target LLVM subset, aggregate
objects are structures, the first index has type \llvm{i32} and value \llvm{0}
to select the first structure at the given location \nt{exp}, and the second
index selects a field in that structure.  Instruction \llvm{load} assigns to a
fresh variable \nt{name} the contents of a memory address of type \nt{type}
specified by \nt{exp} (e.g. in fig.~\ref{fig:ex-llvm}).  Instruction
\llvm{store} writes a value to memory address (again, see
fig.~\ref{fig:ex-llvm}).  Instruction \llvm{br} has two forms.  In the first
form, it directs conditionally the control flow to one of two blocks.  If the
given expression evaluates to \llvm{1}, control goes to the first block,
otherwise to the second. In the second form, the control jumps unconditionally
to the given block.
% Instruction \llvm{switch} directs the
% control flow to one of several blocks. The given expression is evaluated and
% control goes to the first branch with the matching value, or to the default
% branch (appearing first in the instruction).  
Finally, instruction \llvm{ret}
ends the current function call, optionally returning a value.

The expression language is limited to names (local and global), integer literals
and selection of an element in a structure.

\section{Translation: a Bird's-Eye View}
\label{sec:overview}

The translation unit is the B implementation, say \B{M}, and is mapped to a LLVM
module, say \llvm{M}. A B implementation \B{M} has several sections, each
translated as follows:
\begin{enumerate}
\item Visibility clauses provide access to definitions found in external
  B components and are translated to external declarations in LLVM.
\item Constants and value definitions are translated to their value in the
  LLVM platform.
\item To encode the implementation state, the translation produces a
  structure type, named \llvm{\%M\$state\$}, and an instance of that type, named
  \llvm{@M\$self\$}. All the variables and instances of imported machines
  are encoded as elements of that structure.  Also, for each operation of each
  imported machine, an external function declaration is produced.
\item Each operation \B{op} of the implementation is translated into the
  definition of a function named \llvm{@M\$op}; also the translation produces a
  special function \llvm{@M\$init\$} responsible for the initialization.

  The parameter list of the functions thus produced contain one item for each
  input and output of the corresponding operation, and one item of type
  \llvm{\%M\$state\$*}, to pass the address of the structure encoding the
  state of the implementation and its modification by the callee. The return
  type is always \llvm{void}.
\end{enumerate}
This approach assumes that information on external B components used in \B{M} is
available during translation. To fulfill this hypothesis, the translation
produces an ``interface'' artifact containing the information that the
translation of an importation of \B{M} in a client machine
requires. Figure~\ref{fig:skel} summarizes the translation principles by
providing the structure of both the interface and the LLVM implementation thus
produced.

\begin{figure}
  \begin{center}
    (a) Structure of the interface artifact produced from a B implementation \B{M}:
    \begin{tabbing}
      foo \= foo \= \kill
      \textit{(references to imported components)} \\
      \llvm{\%M\$state\$ = type \{ \ListOf{\nt{type}} \}} \\
      \llvm{declare void @M\$init\$(\%M\$state\$*)} \\
      \textit{for each operation} \B{op} \textit{generate} \\
      \> \llvm{declare void @M\$op(\%M\$state\$*, \ListOf{\nt{type}})} \\
    \end{tabbing}
    (b) Structure of the LLVM module produced from a B implementation \B{M}:
    \begin{tabbing}
      foo \= foo \= \kill
      \textit{for each imported component} \B{Q} \textit{generate} \\
      \> \llvm{\%Q\$state\$ = type \{ \ListOf{\nt{type}} \}} \\
      \> \llvm{declare void @Q\$init\$(\%Q\$state\$*)} \\
      \> \textit{for each operation} \B{op} \textit{generate} \\
      \> \> \llvm{declare void @Q\$op(\%Q\$state\$*, \ListOf{\nt{type}})} \\
      \llvm{\%M\$state\$ = type \{ \ListOf{\nt{type}} \}} \\
      \llvm{@M\$self\$ = common global \%M\$state\$ zeroinitializer} \\
      \llvm{define void @M\$init\$(\%M\$state\$* \%self\$) \{} \\
      \> \llvm{\ListOf{\nt{block}}} \\
      \> \llvm{exit: ret void} \\
      \llvm{\}} \\
      \textit{for each operation} \B{op} \textit{generate} \\
      \> \llvm{define void @M\$op(\%M\$state\$* \%self\$, \ListOf{\nt{param}}) \{} \\
      \> \> \llvm{\ListOf{\nt{block}}} \\
      \> \> \llvm{exit: ret void} \\
      \> \llvm{\}}
    \end{tabbing}
  \end{center}
  \caption{Skeleton of the artifacts produced by the translation.}
  \label{fig:skel}
\end{figure}

The detailed presentation of the translation of the different elements of a B
implementation is divided into:
\begin{description}
\item[Modularity] constructs to combine B components are described in
  sec.~\ref{sec:module}.
\item[Data] translation is briefly described sec.~\ref{sec:data}, including
  the encoding of B data types.
\item[Expressions] and conditions are codified as LLVM instruction blocks, as
  described in section~\ref{sec:expr}.
\item[Control] flow, instructions are translated to LLVM instructions as
  discussed in sec.~\ref{sec:control}, which also presents the encoding
  of B operations and initialization.
\end{description}

Code generation is defined with a set of rules defined by structural
recursion. For each grammatical construct, a code generation function is
defined: $\trad{~}{\B{Oper}}$ for operations, $\trad{~}{\B{Inst}}$ for
instructions, $\trad{~}{\B{Expr}}$ for expressions and so forth. Each such
function usually produces a text corresponding to bits of LLVM source code. It
may have additional parameter and results. The symbol $\sep$ is used to clearly
separate composed results. Auxiliary functions are introduced for
conveniency. To abbreviate the presentation, we denote
$\trad{~}{\ListOf{\B{A}}}$ as the translation function for sequences of \B{A}
elements, defined as the concatenation of the application of $\trad{~}{\B{A}}$
to the elements of the sequence.

In those rules, $\PH{t}$ indicates a text place-holder where the value of $t$
shall be inserted and $\nl$ denotes a new line. For instance, we define
$\local(\B{n}) \isdef \llvm{"\%\PH{\B{n.id}}"}$ as a function that given an
element \B{n} of the abstract syntax tree returns the corresponding LLVM local
name, obtained by prepending character \% to its identifier; and $\Global(\B{n})
\isdef \llvm{"@\PH{\B{n.root.id}}\$\PH{\B{n.id}}"}$ yields the corresponding
LLVM global name, composed by special characters (to avoid name conflicts), the
name of the component where it is defined and its own name. Also $\state(\B{M})
\isdef \llvm{"\%\PH{\B{M.id}}\$state\$"}$ yields the name of the type for the
state of component \B{M}. We assume an unlimited supply of labels and local
variables: we use $\lbl$ to return a fresh LLVM label and $\variable$ for a
fresh local variable name.

\section{Translation: Modularity Aspects}
\label{sec:module}

The translation of a B module produces two artifacts: an interface and an (LLVM)
implementation. In the following, we describe how they are produced.

\subsection{Interface}

The definition of the interface $\intf(\B{M})$ of a component \B{M} has four
key elements, as shown in figure~\ref{fig:skel} (b):
\begin{enumerate}
\item $\intf(\B{M})_u$ is the list of used components: $\intf(\B{M})_u \isdef \B{M.imp}$.

\item $\intf(\B{M})_t$ is a textual description of the type representing the
  state of the component, i.e. a structure with one element per imported
  component and variable:
\begin{small}
$$\intf(\B{M})_t \isdef \llvm{"\{\PH{\trad{\B{M.imp}}{\ListOf{\B{Impo}}}_t} \PH{\trad{\B{M.var}}{\ListOf{\B{Vari}}}_t} \}"},$$
\end{small}
where $\trad{\B{i}}{\B{Impo}}_t \isdef \llvm{\state(\B{i})}$ and
$\trad{\B{v}}{\B{Vari}}_t \isdef \trad{\B{v.type}}{\B{Type}}$.

\item $\intf(\B{M})_o$ represents the operation declarations. It maps
operation names to the list of input parameter types and output parameter
types:
$\intf(\B{M})_o \isdef \trad{\B{M.op}}{\ListOf{\B{Oper}}}_t$ and
$\trad{\B{o}}{\B{Oper}}_t \isdef \B{o} \mapsto ( \trad{\B{o.inp}}{\ListOf{inp}} \sep \trad{\B{o.out}}{\ListOf{out}} )$, where
$\trad{\B{i}}{inp} \isdef \trad{\B{i.type}}{\B{Type}}$ and
$\trad{\B{o}}{out} \isdef \llvm{"\PH{\trad{\B{o.type}}{\B{Type}}}*"}$.
\end{enumerate}

\subsection{Implementation}

The actual LLVM source code is generated following the template shown in
figure~\ref{fig:skel} (a), as follows:
\begin{align*}
  \lefteqn{\trad{\B{i}, \intf}{\B{Impl}} \isdef
  \llvm{"\PH{\trad{\B{i.imp}, \intf}{\ListOf{IType}}}} 
  \quad \PH{\trad{\B{i.imp}, \intf}{\ListOf{IInit}}}
  \quad \PH{\trad{\B{i.imp}, \intf}{\ListOf{ICons}}}} \\
& \llvm{\PH{\trad{\B{i.imp}, \intf}{\ListOf{IOper}}}
  \quad \PH{\trad{\B{i.id}, \B{i.const}}{\ListOf{\B{Cons}}}}
  \quad \PH{\trad{\B{i}}{State}}} \\
& \llvm{@\PH{\B{i.id}}\$self\$ = common global \PH{\state(\B{i.id})} zeroinitializer \nl} \\
& \llvm{\PH{\trad{\B{i.init}}{Init}} 
  \quad \PH{\trad{\B{i.op}}{\ListOf{\B{Oper}}}}"}
\end{align*}
We shall now define each rule referenced in this template.
\begin{enumerate}
\item For the type definitions corresponding to the state of the imported
  components: $\trad{i, \intf}{IType} \isdef \llvm{"\PH{\state(\B{i})} =
    type \PH{\intf(\B{i})_t}\nl"}$.
\item For the declarations of the initialization functions in the imported
  components: $\trad{\B{i}, \intf}{Init} \isdef \llvm{"declare void
    @\PH{\B{i.id}}\$init\$(\PH{\state(\B{i.id})}*)\nl"}$.
\item For the declarations of the imported operations: $\trad{\B{i},
    \intf}{IOper} = \trad{\B{i.id}, $\intf(\B{i})_o$}{\ListOf{Oper}}$ where
  $\trad{\B{Q}, $\B{op} \mapsto (i, o)$}{Oper} = \llvm{"declare void
    \PH{\Global(\B{op})}(\PH{\state(\B{Q})}*,\PH{i},\PH{o})"}$.
\item The translation of constant definitions is presented in section~\ref{sec:data}.
\item For the definition of the type for component states: \\
\noindent$\trad{\B{M}}{State} \isdef
\llvm{"\PH{\state(\B{M})} = type \PH{\intf(\B{M})_t}\nl"}$.
\item For the initialization, the definition of a function is generated. It
  first initializes the imported modules and then the current module (see
  section~\ref{sec:control}).
\item For each operation, a function definition is generated (see
  section~\ref{sec:control}).
\end{enumerate}
Let $\idx$ be a function that maps each imported machine and state variable to
its position (starting from zero).
\section{Translation: Data}
\label{sec:data}

The translation presented in this paper is currently limited to integers and
booleans. In the case of integers, the translation is based on the supposition
that the B0 definition of \B{INT} is the range $-2^{31}.. 2^{31}-1$. We have
$\trad{BOOL}{\B{Type}} \isdef \llvm{i1}$ and $\trad{INT}{\B{Type}} \isdef
\llvm{i32}$.

B constants shall be translated to constants in the LLVM program. The
translation of scalar constants consists in substituting them with their value.
\DD{The translation of non-scalar constants will be defined in a next version of
  this document.}  The definition of the translation for expressions is
described next in section~\ref{sec:expr}.

\section{Translation: Expressions}
\label{sec:expr}

To evaluate a composed expression using the LLVM, one must first generate a
sequence of instructions to evaluate each argument, the resulting value being
stored in a temporary variable, and second generate an instance of the LLVM
instruction corresponding to the expression argument.  Translation of
expressions is defined as functions that take as input some class of expressions
and produce a triple of strings $p \sep v \sep t$, where $p$ is a possibly empty
sequence of instructions required in the evaluation (e.g.  for the
sub-expressions), $v$ is the LLVM expression representing the resulting value,
and $t$ represents the LLVM type of the expression. In the case of composed
predicates, though, the evaluation is different as boolean operations need to be
implemented with conditional and unconditional jumps.

To facilitate the defintiion of rules that apply to similar cases, we introduce
mapping $\opmap$, that associates B operators to corresponding LLVM keywords:
$$\opmap = \{ 
\begin{array}[t]{l}
  = \mapsto \llvm{"eq"}, 
  \ne \mapsto \llvm{"ne"}, 
  < \mapsto \llvm{"slt"}, 
  \le \mapsto \llvm{"sle"}, 
  > \mapsto \llvm{"sgt"}, 
  \ge \mapsto \llvm{"sge"}, \\
  + \mapsto \llvm{"add"} \}.
  \end{array}
$$

\paragraph{Atomic expressions} have a simple translation. In the case of
literals, one only needs to issue the corresponding LLVM literal:
$\trad{TRUE}{\B{Lit}} \isdef \llvm{""} \sep \llvm{"1"} \sep \llvm{"i1"}$,
$\trad{FALSE}{\B{Lit}} \isdef \llvm{""} \sep \llvm{"0"} \sep \llvm{"i1"}$ and
$\trad{l}{\B{Lit}} \isdef \llvm{""} \sep \llvm{"\PH{l}"} \sep \llvm{"i32"}$ when
$\B{l.type} = \B{INT}$.

For identifiers, three cases are possible: global constants, local variables,
and module state variables. For state variables, the translation issues
statements to assign the address of the corresponding element in the state
structure to pointer variable $p$ and then dereference $p$ to assign $v$:
\begin{align*}
\lefteqn{\trad{n}{\B{Name}} \isdef \LET t = \trad{n.type}{\B{Type}} \IN} \\
& \IF \B{n} \mbox{ is a constant } \THEN \trad{n.value}{\B{Expr}} \\
& \ELSIF \B{n} \mbox{ is a local variable } \THEN 
\llvm{""} \sep \local(\B{n}) \sep t \\
& \ELSE (\B{n} \mbox{ is a state variable }) \LET p = \variable \AND v = \variable \IN \\
& \quad \llvm{"\PH{p} = getelementptr \PH{\state(\B{n.root})} \%self\$, i32 0, i32 \PH{\idx(\B{n})} \nl} \\
& \quad \llvm{\PH{v} = load \PH{t}* \PH{p}\nl"} \sep v \sep t
\end{align*}

\paragraph{Comparisons} are formed by the application of a relational operator
to two expressions. 

The code responsible for evaluating a comparison \B{r} is defined by the
following rule. First both arguments are evaluated in $p_1$ and $p_2$, and then
an instance of the LLVM comparison instruction \llvm{icmp} is used to compare
the temporary variables $v_1$ and $v_2$, of type $t_1 = t_2$, storing the result
in the fresh variable $v$.
\begin{align*}
\lefteqn{\trad{r}{\B{Comp}} \isdef
  \LET
  p_1 \sep v_1 \sep t_1 = \trad{r.arg1}{\B{Expr}} \AND
  p_2 \sep v_2 \sep t_2 = \trad{r.arg2}{\B{Expr}}} \\
& \IN \LET v = \variable \IN \\
& \llvm{"\PH{p_1}\quad\PH{p_2}\quad\PH{v} = icmp \PH{\opmap(\B{r.op})} \PH{t_1} \PH{v_1} \PH{v_2} \nl"} \sep v
\end{align*}
   
\paragraph{Arithmetic operations}\DD{Actually some arithmetic operations are not binary and different translation rules shall be written.} have a similar 
structure and interpretation as comparisons. Here is the rule for binary
arithmetic operations:
\begin{align*}
\lefteqn{\trad{r}{\B{Term}} \isdef
  \LET
  p_1 \sep v_1 \sep t_1 = \trad{r.arg1}{\B{Expr}} \AND
  p_2 \sep v_2 \sep t_2 = \trad{r.arg2}{\B{Expr}}} \\
& \IN \LET v = \variable \IN \\
& \llvm{"\PH{p_1}\quad\PH{p_2}\quad\PH{v} = i32 \PH{\opmap(\B{r.op})} \PH{t_1} \PH{v_1} \PH{v_2} \nl"} \sep v \sep \llvm{i32}
\end{align*}
   
Next is the translation of the unary \B{succ} operator:
\begin{align*}
  \trad{n}{\B{succ}} \isdef & \textbf{ let } p \sep v \sep t = \trad{n.arg}{\B{expr}} \IN  \\
  & \quad \LET w = \variable \IN \\
  & \quad \quad \llvm{"\PH{p}} \\
  & \quad \quad \llvm{\PH{w} = add i32 1, \PH{v} \nl"} \sep w \sep \llvm{i32}
\end{align*}

\paragraph{Predicates}
The translation of conditions takes as input a condition node \B{n}, as well as
two labels $\ell_t$ and $\ell_f$ that correspond to program locations where the
execution shall jump when the condition is evaluted to true or false,
respectively.  The first rule defines the translation of an atomic relation:
\begin{align*}
  \lefteqn{\trad{n, $\ell_t$, $\ell_f$}{\B{Form}} \isdef
  \textbf{ let } p \sep v = \trad{n}{\B{Comp}} \IN} \\
  & \llvm{"\PH{p} \quad br i1 \PH{v}, label \%\PH{\ell_t}, label \%\PH{\ell_f} \nl"}
\end{align*}

\noindent The translation for negation is simply: $\trad{not f, $\ell_t$,
  $\ell_f$}{\B{not}} \isdef \trad{f, $\ell_f$, $\ell_t$}{\B{Form}}$.  The
following rule handles the case of conjunctions. Note that it produces code with
a ``short-cut'' when the evaluation of the first argument yields false:
\begin{align*}
\begin{split}
  \lefteqn{\trad{n, $\ell_t$, $\ell_f$}{\B{and}} \isdef \LET \ell = \lbl \IN} \\
  & \LET p_1 = \trad{n.arg1, $\ell$, $\ell_f$}{\B{Form}}
  \AND p_2 = \trad{n.arg2, $\ell_t$, $\ell_f$}{\B{Form}} \IN \\
  & \quad \llvm{"\PH{p_1} \PH{\ell} :  \nl \PH{p_2}"}
\end{split}
\end{align*}
The definition of the translation for disjunctions is similar and is omitted.

\section{Translation: Control Flow and Instructions
\label{sec:control}}

This section addresses how operations in B implementations are translated to
LLVM functions. It also discusses initialization, which might be seen as a
special kind of operation. For operations, the presentation of the translation
is divided into three steps: the signature, the local data, and the body
instructions.

The signature of the function is composed of the name of the function, the
result type, the name and type of the parameters. The signature may carry other
information such as linkage, visibility, calling convention, etc. The B parser
provides this information after the type checking phase has been completed.  On
the one hand LLVM functions may only have one result, on the other hand B0
operations may produce several results. The translation of output parameters
will be function parameters passed by reference, i.e. they are pointers. For
uniformity of treatment, the return type of the produced LLVM functions is
\llvm{void}. Translation of operation headers is defined in
sec.~\ref{sec:trad-header}. The memory to store local variables is allocated on
the stack, using the \llvm{alloca} instruction. We describe a function in
sec.~\ref{sec:trad-alloc} that, applied to a B operation, returns the name and
type of its local variables used in the operation. These names are translated to
conform to LLVM restrictions. The body of the operation has to be transformed to
a sequence of LLVM blocks of statements. For each kind of instruction one (or
more) translation function is defined. The details are provided in
section~\ref{sec:trad-instr}.  The results of the translation of each aspect are
combined as follows:
\begin{align*}
\begin{split}
  \lefteqn{\trad{\B{op}}{\B{oper}} \isdef} \\
  & \LET h = \trad{\B{op}}{sig} \AND   a = \trad{\B{op.body}}{alloc} \AND i = \trad{\B{op.body}, \llvm{exit}}{\ListOf{\B{Inst}}}_L \IN  \\
  & \llvm{"define void \PH{h} \{\nl} \\
  & \quad \llvm{entry:\nl} \\
  & \quad \quad \PH{a} \\
  & \quad \quad \PH{i} \\
  & \quad \llvm{exit: ret void\nl} \\
  & \llvm{\}\nl"}
\end{split}
\end{align*}

Initialization of a B implementation shares many characteristics with operations
but needs to deal with initialization of imported modules. It is discussed in
sec.~\ref{sec:trad-init}.

\subsection{Sub-routines signatures}
\label{sec:trad-header}

The following rule specifies the translation of operation headers. Its
parameters are the operation identifier, inputs, outputs, the name of the
machine it belongs to, and the results are the LLVM function definition
header: \\
\noindent$\trad{\B{o}}{sig} \isdef \llvm{"\PH{\Global(\B{o})}(\PH{\state(\B{o.root})}* \%self\$, \PH{\trad{\B{o.inp}}{\ListOf{inp}}}, \PH{\trad{\B{o.out}}{\ListOf{out}}})"}$

Next are the rules defining the translation of individual input and output
parameters. They have local scope and they are translated by prepending
\llvm{\%}:
\begin{align*}
  \trad{i}{inp} \isdef 
  \llvm{"\PH{\trad{i.type}{\B{Type}}} \PH{\local(\B{i})}"} \quad \quad
  \trad{o}{out} \isdef 
  \llvm{"\PH{\trad{o.type}{\B{Type}}}* \PH{\local(\B{o})}"}
\end{align*}

\subsection{Stack allocation}
\label{sec:trad-alloc}

B operations may have local variables, in which case the translator needs to
issue LLVM stack allocation instructions. Such instructions have two arguments:
the type of values to store, and a temporary variable that is assigned the
address (see figure~\ref{fig:llvm-grammar} for the concrete syntax). The
generation of stack allocation code visits the parse tree of the operation body
in a pre-order depth-first traversal, generating one statement for each local
variable declaration found. The detailed definition of this traversal is omitted
for space reasons. For each B0 variable \B{v} found, a LLVM name is thus
created.
\begin{align*}
  \trad{v}{alloc} \isdef 
  \llvm{"\PH{\local(\B{v})} = alloca \PH{\trad{v.type}{\B{Type}}} \nl"}
\end{align*}

After the stack allocations, the LLVM source code contains the statements that
encodes the instructions in the body of the operation, as described in the
following section.

\subsection{Translation of instructions}
\label{sec:trad-instr}

One issue in the translation of instructions is the disruption of the flow graph
intro a linear sequence of instructions and branches. This requires the creation
of instruction labels. As a consequence, our definition of the translation of
instructions has two kinds of rules, those that take a label as parameter, and
those that do not. Such label parameter indicates where the control should go
after the corresponding execution is executed.

\begin{align*}
\trad{\B{i}, $\ell$}{\B{Inst}}_L \isdef
& \IF \B{i} \mbox{ is \B{If}} \THEN \trad{\B{i}, $\ell$}{\B{If}}_L \\
& \ELSIF \B{i} \mbox{ is \B{While}} \THEN \trad{\B{i}, $\ell$}{\B{While}}_L \\
& \ELSIF \B{i} \mbox{ is \B{Blk}} \THEN \trad{\B{i.body}, $\ell$}{\ListOf{\B{Inst}}}_L \\
& \ELSE \LET p = \trad{\B{i}}{\B{Inst}} \IN \\
& \quad \llvm{"\PH{p}} \\
& \quad \llvm{ br label \%\PH{\ell} \nl"}
\end{align*}

Function $\trad{\B{il}, $\ell$}{\ListOf{\B{Inst}}}_L$ takes as input a sequence
of B instructions \B{il} and a label $\ell$ and produces the source code of a
LLVM block with $\ell$ as exit point. It is defined as follows: 

\begin{align*}
\trad{\B{il}, $\ell$}{\ListOf{\B{Inst}}}_L \isdef & \IF \B{il} \mbox{ is empty} \THEN \llvm{"br label \%\PH{\ell} \nl"} \\
& \ELSE \\
& \LET \B{i}, \B{il'} = \B{il}  \IN \\
& \quad \quad \LET p_1 = 
\begin{array}[t]{l}
  \IF \B{i} \mbox{ is \B{If} or \B{While}} \THEN \\
  \quad \LET \ell' = \lbl \IN \llvm{"\PH{\trad{i, $\ell'$}{\B{Inst}}_L} \PH{\ell'}: \PH{p}"} \\
  \ELSE \llvm{"\PH{\trad{i}{\B{Inst}}} \PH{p}"} \IN
\end{array} \\
& \quad \quad \IN \\
& \quad \quad \quad \LET p_2 = \trad{il', $\ell$}{\ListOf{\B{Inst}}}_L \\
& \quad \quad \quad \quad \llvm{"\PH{p_1} \PH{p_2}"}
\end{align*}

Eventually, $\trad{}{\B{Inst}}$ defines the translation for individual
instructions. Its definition is specialized for each class of instructions as
detailed in the rest of this section. Note that a new block is created after
each \B{If} and \B{While} instruction.

\paragraph{While instructions} are translated in two blocks of instructions:
block $\ell_1$ evaluates the loop condition and block $\ell_2$ executes the loop
body:
\begin{align*}
\begin{split}
\trad{w, $\ell$}{\B{While}} \isdef 
  & \LET \ell_1 = \lbl \AND \ell_2 = \lbl \AND c \sep v = \trad{w.cond}{\B{Pred}} \IN \\
  & \llvm{"br label \%\PH{\ell_1} \nl} \\
  & \llvm{\PH{\ell_1} : \PH{c}} \\
  & \llvm{br i1 \PH{v}, label \%\PH{\ell_2}, label \%\PH{\ell} \nl} \\
  & \llvm{\PH{\ell_2} : \PH{\trad{i.body, $\ell_1$}{\ListOf{\B{Inst}}}_L}}
\end{split}
\end{align*}

\paragraph{If instructions} are sequences of possibly conditional blocks of
instructions, i.e. branches. After a branch is executed, control must go
to the next block $\ell$.
\begin{align*}
  \trad{i, $\ell$}{\B{If}}_L \isdef \trad{i.branches, $\ell$}{\ListOf{\B{IfBr}}}
\end{align*}
To translate conditional branches, one must first generate code to evaluate the
condition of the branch, followed by an LLVM conditional branching statement,
translate the body of the branch, and add an unconditional branch to the block
following the corresponding \B{If} statement.  The translation for branches has
as arguments a branch \B{b}, the remaining branches \B{bl}, the lexicon, and the
label $\ell$ of the block following the \B{If} statement they belong to. The
following definition applies for the case of conditional branches:
\begin{align*}
\begin{split}
  \trad{\B{b bl}, $\ell$}{\ListOf{\B{IfBr}}} \isdef & \LET \ell_1 = \lbl \AND \ell_2 = \lbl \IN \\
  & \llvm{"\PH{\trad{b.cond, $\ell_1$, $\ell_2$}{\B{Pred}}} } \\
  & \llvm{\PH{\ell_1} : \PH{\trad{b.body, $\ell$}{\ListOf{\B{Inst}}}_L}} \\
  & \llvm{\PH{\ell_2} : \PH{\trad{bl, $\ell$}{\ListOf{\B{IfBr}}}}"} \\
\end{split}
\end{align*}
The following rule handles the case of the last branch, which might have a condition or not:
\begin{align*}
\begin{split}
  \trad{\B{b}, $\ell$}{\ListOf{\B{IfBr}}} \isdef & \IF \B{b.cond} = \text{None} \THEN
  \trad{b.body, $\ell$}{\ListOf{Inst}}_L\\
  & \ELSE \LET \ell_1 = \lbl \IN \\
  & \llvm{"\PH{\trad{b.cond, $\ell_1$, $\ell$}{\B{Pred}}} } \\
  & \llvm{\PH{\ell_1} : \PH{\trad{b.body, $\ell$}{\ListOf{\B{Inst}}}_L}}
\end{split}
\end{align*}

\paragraph{Becomes equal instructions} must evaluate the target and the source
of the assignment, and then copy the result of the latter in the former, using
the LLVM instruction \llvm{store}. The evaluation of the target is defined with
$\trad{}{lv}$, which yields the corresponding LLVM source code $l$ and the
variable $p$ containing the assigned location. The evaluation of the source
yields the corresponding code $r$, the variable $v$ holding the result, and its
type $t$.
\begin{align*}
\begin{split}
  \trad{a}{\B{Beq}} \isdef 
  & \LET l \sep p = \trad{i.lhs}{lv} \AND r \sep v \sep t = \trad{i.rhs}{\B{Expr}} \IN \\
  & \llvm{"\PH{l}} \\
  & \llvm{\PH{r}} \\
  & \llvm{store \PH{t} \PH{v}, \PH{t}* \PH{p} \nl"}
%   & l \conc r \conc 
%   \llvm{"store "} \conc t \conc \texttt{" "} \conc v \conc \llvm{", "}
%   \conc t \conc \llvm{"* "} \conc p \conc \nl
\end{split}
\end{align*}
The assignment target may be either local to the operation or a state
variable. In the former case, it can be referenced by the corresponding LLVM
identifier. In the latter case, it is represented as an element of the state
structure $\Sigma$, and its address must be calculated and assigned to a new temporary
variable.
\begin{align*}
\lefteqn{\trad{n}{lv} \isdef} \\
& \quad \IF \mbox{\B{n} is a local variable} \THEN \llvm{""}, \local(\B{n}) \\
& \quad \ELSE \LET v = \variable \AND t = \trad{n.type}{\B{Type}} \IN \\
& \quad \quad \llvm{"\PH{v} = getelementptr \PH{\state(\B{n.root})}* \%self\$,} \\
& \quad \quad \quad \llvm{i32 0, i32 \PH{\idx(\B{n})} \nl"} \sep v
\end{align*}

\paragraph{Call up} instructions apply B operations with some values and obtain
results. The LLVM code generated must include instructions to evaluate the
operation arguments, the address of the variables receiving the results as well
as an instance of the LLVM \llvm{call} instruction. In addition, the proposed
translation includes as argument the address of a structure storing the state of
the component corresponding to this operation. Note that this component may be
either the implementation being translated, or an instance of an imported
machine.
\begin{align*}
  \trad{c}{\B{Call}} \isdef 
  & \LET p_i \sep il = \trad{c.inp}{\ListOf{Inp}} \AND p_o \sep ol = 
  \trad{c.out}{\ListOf{Out}} \AND \\
  & \quad id = \IF \B{c.pre} = \text{None} \THEN \Global(\B{c}) \ELSE \llvm{"@\PH{c.pre}\$\PH{c.id}"} \AND \\
  & \quad ty = \IF \B{c.pre} = \text{None} \THEN \state(\B{c.root}) \ELSE \llvm{"\%\PH{c.pre}\$state\$"} \AND \\
  & \quad val = \IF \B{c.pre} = \text{None} \THEN \llvm{"\%self\$"} \ELSE \\
  & \quad \quad \llvm{"getelementptr \PH{\state(\B{n.root})} \%self\$, i32 0, i32 \PH{\idx(\B{c.pre})}"}\\
  & \IN \\
  & \quad \llvm{"\PH{p_i}} \\
  & \quad \llvm{\PH{p_o}} \\
  & \quad \llvm{call void \PH{id}(\PH{ty}* \PH{val}, \PH{il}, \PH{ol}) \nl"}
\end{align*}
The translation of input and output arguments produces two code snippets. The
first corresponds to the LLVM instruction sequence $p$ responsible for
evaluating the argument, and the second to the type $t$ and name $v$ of the
variable storing the resulting value:
\begin{align*}
  \trad{i}{Inp} \isdef 
  & \LET p \sep v \sep t = \trad{i}{\B{Expr}} \IN p \sep \llvm{"\PH{t} \PH{v}"}\\
  \trad{o}{Out} \isdef 
  & \LET p \sep v \sep t = \trad{o}{\B{Expr}} \IN p \sep \llvm{"\PH{t}* \PH{v}"}
\end{align*}

\paragraph{Block instructions} are structuring constructs that have no
operational intent; their translation is that of their body: $\trad{i,
  $\ell$}{\B{Blk}} \isdef \trad{i.body, $\ell$}{\ListOf{\B{Inst}}}$.


\subsection{Initialization}
\label{sec:trad-init}

Initialization is done is two steps: first all imported modules are initialized,
second the initialization clause is executed.
\begin{align*}
  \trad{i}{Init} \isdef
  & \llvm{"define void @\PH{\B{i.id}}\$init\$() \{ \nl"} \\
  & \quad\llvm{\PH{\trad{i.impl}{\ListOf{InitImp}}}} \\
  & \quad\llvm{\PH{\trad{i.init, "exit"}{\ListOf{\B{Inst}}}_L}} \\
  & \quad\llvm{exit: ret void \nl} \\
  & \llvm{\}"}
\end{align*}
The initialization of an imported module is performed by a call to the
corresponding initialization function:
\begin{align*}
  \lefteqn{\trad{m}{InitImp} \isdef } \\
  & \LET \ell = \lbl \IN \\
  & \quad \llvm{"\PH{\ell} = getelementptr \PH{\state(\B{m.root})}* \%self\$, i32 0, i32 \PH{\idx(\B{m})}\nl}\\
  & \quad \llvm{call void @\PH{\B{m.id}}\$init\$(\PH{\state(\B{m})}*
    \PH{\ell})"}
\end{align*}

\section{Validation}
\label{sec:validation}

Before implementing this translation, we want validate the rules that define it.
For this, we follow a pragmatic approach, based on testing and simulation of the
generated LLVM artifacts, and using the existing LLVM tool set.  

First, we define a sample set of small B implementation modules, that contain
all the B constructs handled in the translation. For each such module, the rules
are applied manually, yielding a LLVM module. This LLVM module is then amended
with extra code that have two purposes:
\begin{itemize}
\item a standalone executable may be generated from it;
\item the executable creates an instance of the B implementation, applies the
  initialization code, and offers to the user the choice to interactively invoke
  all the operations of the implementation; after initialization and each
  operation application the state of the implementation is printed. 
\end{itemize}

The LLVM tools that are employed are:
\begin{description}
\item[llvm-as] The LLVM assembler generates LLVM bitcode, invoked, e.g., as
  follows:
\begin{verbatim}
llvm-as-mp-3.1 example-01.llvm -o example-01.bc
\end{verbatim}
\item[lli] The LLVM bitcode interpreter, which may be invoked as follows:
\begin{verbatim}
lli-mp-3.1 example-01.bc
\end{verbatim}
\end{description}

Once a rule has been manually validated, we implement it in a prototype tool
written in Python. Unfortunately the tool does not take as input the B0 source
file, but instead a collection of Python definitions corresponding to the
syntactic entities present in the parse tree. For instance, for an integer
concrete variable named \B{value}, we would have the following Python entity:
\begin{verbatim}
value = { "kind": "Vari",
          "id": "value",
          "type": "INT",
          "scope": "impl",
          "root": counter_i}
\end{verbatim}
where \verb'counter_i' would be the Python entity representing the parse tree
root for the B implementation. Again the output generated by the prototype
is compiled and simulated using off-the-shelf LLVM tools \verb'llvm-as' and
\verb'lli'.

\section{Conclusion and future work}
\label{sec:conclusion}

This paper presents the formal specification of a translation from a large
subset of the B implementation language to LLVM, a modern compiler internal
language.  Even though this is still a work in progress, the definition is
self-contained and has a large enough scope to be applied to B implementations
where the data belongs to basic types. This specification is a blueprint for a
code synthesis tool for B that is currently being implemented. This tool will be
distributed as an extension to Atelier-B under an open-source license.

The next step is to extend the scope to the full B implementation language. This
entails that the translation must be defined for aggregate data types and
adapted to handle some syntactic sugar. We are also planning for producing a
LLVM output with debugging information. Such output would be indeed very helpful
to provide feedback to the user when applying testing to validate the produced
code.

To prove the verification of the correctness of the translation, we would have
to define the semantics of both LLVM and B in a unified framework. One possible
starting point is Vellvm~\cite{vellvm}, a framework to reason about the
correctness of LLVM programs and transformations. We would then have to extend
the framework to include B implementation language. Another possible approach
would be to translate verification conditions from the B development artifacts
as assertinons in the generated LLVM code. The compiled program would include
checks that such assertions hold while executing.

\bibliographystyle{plain}
\bibliography{b2llvm}

\newpage
\appendix

\section{Test summary}

For each test file, indicates the B0 constructs found in the implementation.
\begin{center}
\newcommand{\no}[0]{--}
\newcommand{\yes}[0]{$\circ$}
\begin{tabular}{|l|c|c|c|c|c|}
\hline
Example nº & 01 & 02 & 03 & 04 & 05\\
\hline
variables & \yes & \yes & \yes & \yes & \yes\\
constants & \no & \yes & \no & \no & \yes\\
op. input & \yes & \no & \yes & \no & \no \\
op. output & \yes & \no & \yes & \no & \no \\
\hline
\multicolumn{6}{|c|}{instructions} \\
\hline
block & \yes & \no & \yes & \yes & \no\\
local var & \no & \no & \yes & \yes & \no \\
identity & \no & \no & \yes & \no & \no \\
becomes equal & \yes & \yes & \yes & \yes & \yes\\
operation call & \no & \no & \no & \no & \no \\
condition & \yes & \no & \no & \no & \yes\\
case & \no & \no & \no & \no & \no \\
assertion & \no & \no & \no & \no & \no \\
sequence & \yes & \yes & \yes & \yes & \no \\
while & \no & \no & \no & \yes & \no \\
\hline
\multicolumn{6}{|c|}{expressions} \\
\hline
int lit & \yes & \yes & \yes & \yes & \yes\\
bool lit & \yes & \yes & \no & \yes & \yes\\
vars & \yes & \yes & \yes & \yes & \no \\
loc. vars & \no & \no & \yes & \yes & \no \\
const & \no & \yes & \no & \no & \yes\\ 
succ & \no & \yes & \no & \yes & \yes\\
pred & \no & \yes & \no & \yes & \yes\\
+ & \yes & \no & \no & \yes & \no \\
- & \yes & \no & \no & \no & \no\\
\hline
\multicolumn{6}{|c|}{formulas} \\
\hline
and & \no & \no & \no & \no & \yes\\
or & \no & \no & \no & \no & \yes\\
not & \no & \no & \no & \no & \yes\\
$<$ & \yes & \yes & \no & \no & \no\\
$>$ & \no & \no & \no & \no & \yes \\
\hline
\end{tabular}
\end{center}


\end{document}

