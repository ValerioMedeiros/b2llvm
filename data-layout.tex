\begin{verbatim}
; ModuleID = 'pg2.c'
target datalayout = "e-p:64:64:64-i1:8:8-i8:8:8-i16:16:16-i32:32:32-i64:64:64-f32:32:32-f64:64:64-v64:64:64-v128:128:128-a0:0:64-s0:64:64-f80:128:128-n8:16:32:64-S128"
target triple = "x86_64-apple-macosx10.7.4"
\end{verbatim}

Data layout of the target architecture

\begin{description}
\item[e] Specifies that the target lays out data in little-endian form. That is, the bits with the least significance have the lowest address location.
\item[p:64:64:64] 64-bit pointers with 64-bit alignment
\item[i1:8:8] i1 is 8-bit (byte) aligned
\item[i8:8:8] i8 is 8-bit (byte) aligned
\item[i16:16:16] i16 is 16-bit aligned
\item[i32:32:32] i32 is 32-bit aligned
\item[i64:32:64] i64 has ABI alignment of 32-bits but preferred alignment of 64-bits
\item[f32:32:32] float is 32-bit aligned
\item[f64:64:64] double is 64-bit aligned
\item[v64:64:64] 64-bit vector is 64-bit aligned
\item[v128:128:128] 128-bit vector is 128-bit aligned
\item[a0:0:64] aggregates are 64-bit aligned
\item[s0:64:64] stack objects are 64-bit aligned
\item[f80:128:128] alignment for a floating point type of a given bit
  size. Only values of size that are supported by the target will
  work. 32 (float) and 64 (double) are supported on all targets; 80 or
  128 (different flavors of long double) are also supported on some
  targets.
\item[n8:16:32:64] native integer widths for the target CPU in
  bits. Elements of this set are supposed to support most arithmetic
  operations efficiently.  S128
\end{description}

Global variables define regions of memory allocated at compilation
time instead of run-time. Global variables may optionally be
initialized, may have an explicit section to be placed in, and may
have an optional explicit alignment specified. 

\begin{lstlisting}
unsigned short c1, c2, c3, c4;
\end{lstlisting}

\begin{verbatim}
@c1 = common global i16 0, align 2
@c2 = common global i16 0, align 2
@c3 = common global i16 0, align 2
@c4 = common global i16 0, align 2
\end{verbatim}
An explicit alignment may be specified for a global, which must be a
power of 2. If not present, or if the alignment is set to zero, the
alignment of the global is set by the target to whatever it feels
convenient. If an explicit alignment is specified, the global is
forced to have exactly that alignment. Targets and optimizers are not
allowed to over-align the global if the global has an assigned
section. In this case, the extra alignment could be observable: for
example, code could assume that the globals are densely packed in
their section and try to iterate over them as an array, alignment
padding would break this iteration.

\begin{lstlisting}
unsigned short f1 (void) {
\end{lstlisting}

\begin{verbatim}
define zeroext i16 @f1() nounwind uwtable ssp {
\end{verbatim}
\begin{description}
\item[zeroext]
    This indicates to the code generator that the parameter or return value should be zero-extended to the extent required by the target's ABI (which is usually 32-bits, but is 8-bits for a i1 on x86-64) by the caller (for a parameter) or the callee (for a return value).
\item[nounwind]
    This function attribute indicates that the function never returns with an unwind or exceptional control flow. If the function does unwind, its runtime behavior is undefined.
\item[uwtable]
    This attribute indicates that the ABI being targeted requires that an unwind table entry be produce for this function even if we can show that no exceptions passes by it. This is normally the case for the ELF x86-64 abi, but it can be disabled for some compilation units.
\item[ssp]
    This attribute indicates that the function should emit a stack smashing protector. It is in the form of a "canary"---a random value placed on the stack before the local variables that's checked upon return from the function to see if it has been overwritten. A heuristic is used to determine if a function needs stack protectors or not.
    If a function that has an ssp attribute is inlined into a function that doesn't have an ssp attribute, then the resulting function will have an ssp attribute.
\end{description}

\begin{lstlistings}
  c1 = 96;
  c2 = 128;
\end{lstlisting}
\begin{verbatim}
  store i16 96, i16* @c1, align 2
  store i16 128, i16* @c2, align 2
\end{verbatim}

There are two arguments to the ``store'' instruction: a value to store
and an address at which to store it. The type of the ``pointer''
operand must be a pointer to the first class type of the ``value''
operand. The optional constant ``align'' argument specifies the
alignment of the operation (that is, the alignment of the memory
address). A value of 0 or an omitted ``align'' argument means that the
operation has the preferential alignment for the target. It is the
responsibility of the code emitter to ensure that the alignment
information is correct. Overestimating the alignment results in an
undefined behavior. Underestimating the alignment may produce less
efficient code. An alignment of 1 is always safe.
\begin{verbatim}
store [volatile] <ty> <value>, <ty>* <pointer>[, align <alignment>][, !nontemporal !<index>]
\end{verbatim}

\begin{lstlistings}
  if (c1 < c2) {
\end{lstlistings}

The values stored in variables \lstinline{c1} and \lstinline{c2} are loaded, promoted to integers and compared:
\begin{verbatim}
  %1 = load i16* @c1, align 2
  %2 = zext i16 %1 to i32
  %3 = load i16* @c2, align 2
  %4 = zext i16 %3 to i32
  %5 = icmp slt i32 %2, %4
\end{verbatim}

Unnamed temporaries are created when the result of a computation is not assigned to a named value.
Unnamed temporaries are numbered sequentially

Overview:

The 'load' instruction is used to read from memory.
Arguments:

The argument to the 'load' instruction specifies the memory address from which to load. The pointer must point to a first class type. 

The optional constant align argument specifies the alignment of the operation (that is, the alignment of the memory address). A value of 0 or an omitted align argument means that the operation has the preferential alignment for the target. It is the responsibility of the code emitter to ensure that the alignment information is correct. Overestimating the alignment results in undefined behavior. Underestimating the alignment may produce less efficient code. An alignment of 1 is always safe.

The location of memory pointed to is loaded. If the value being loaded is of scalar type then the number of bytes read does not exceed the minimum number of bytes needed to hold all bits of the type. For example, loading an i24 reads at most three bytes. When loading a value of a type like i20 with a size that is not an integral number of bytes, the result is undefined if the value was not originally written using a store of the same type.

  <result> = zext <ty> <value> to <ty2>             ; yields ty2

The 'zext' instruction zero extends its operand to type ty2.

The 'zext' instruction takes a value to cast, and a type to cast it to. Both types must be of integer types, or vectors of the same number of integers. The bit size of the value must be smaller than the bit size of the destination type, ty2.

Semantics:

The zext fills the high order bits of the value with zero bits until it reaches the size of the destination type, ty2.

When zero extending from i1, the result will always be either 0 or 1.

  <result> = icmp <cond> <ty> <op1>, <op2>   ; yields {i1} or {<N x i1>}:result

The 'icmp' instruction returns a boolean value or a vector of boolean values based on comparison of its two integer, integer vector, pointer, or pointer vector operands.
Arguments:

The 'icmp' instruction takes three operands. The first operand is the condition code indicating the kind of comparison to perform. It is not a value, just a keyword. The possible condition code are:

    eq: equal
    ne: not equal
    ugt: unsigned greater than
    uge: unsigned greater or equal
    ult: unsigned less than
    ule: unsigned less or equal
    sgt: signed greater than
    sge: signed greater or equal
    slt: signed less than
    sle: signed less or equal

  br i1 %5, label %6, label %10

  br i1 <cond>, label <iftrue>, label <iffalse>

\begin{lstlistings}
    c3 = c1;
    c1 = c2;
    c2 = c3;
\end{lstlistings}

\begin{verbatim}
; <label>:6                                       ; preds = %0
  %7 = load i16* @c1, align 2
  store i16 %7, i16* @c3, align 2
  %8 = load i16* @c2, align 2
  store i16 %8, i16* @c1, align 2
  %9 = load i16* @c3, align 2
  store i16 %9, i16* @c2, align 2
  br label %10
\end{verbatim}

\begin{lstlistings}
  }
  c4 = 0;
  while (c1 /= c2) {
    c3 = c1 - c2;
    c1 = c2;
    c2 = c3;
    c4 = c4 + 1;
  }
  return c4;
}
\end{lstlistings}



; <label>:10                                      ; preds = %6, %0
  store i16 0, i16* @c4, align 2
  br label %11

; <label>:11                                      ; preds = %19, %10
  %12 = load i16* @c2, align 2
  %13 = zext i16 %12 to i32
  %14 = load i16* @c1, align 2
  %15 = zext i16 %14 to i32
  %16 = sdiv i32 %15, %13
  %17 = trunc i32 %16 to i16
  store i16 %17, i16* @c1, align 2
  %18 = icmp ne i16 %17, 0
  br i1 %18, label %19, label %32

; <label>:19                                      ; preds = %11
  %20 = load i16* @c1, align 2
  %21 = zext i16 %20 to i32
  %22 = load i16* @c2, align 2
  %23 = zext i16 %22 to i32
  %24 = sub nsw i32 %21, %23
  %25 = trunc i32 %24 to i16
  store i16 %25, i16* @c3, align 2
  %26 = load i16* @c2, align 2
  store i16 %26, i16* @c1, align 2
  %27 = load i16* @c3, align 2
  store i16 %27, i16* @c2, align 2
  %28 = load i16* @c4, align 2
  %29 = zext i16 %28 to i32
  %30 = add nsw i32 %29, 1
  %31 = trunc i32 %30 to i16
  store i16 %31, i16* @c4, align 2
  br label %11

; <label>:32                                      ; preds = %11
  %33 = load i16* @c4, align 2
  ret i16 %33
}


Level1_instruction ::= Block_instruction
| Var_instruction
| Identity_substitution
| Becomes_equal_instruction
| Callup_instruction
| If_instruction
| Case_instruction
| Assert_instruction
| While_substitution

Block_instruction ::=
"BEGIN" Instruction "END"

Var_instruction ::=
"VAR" Ident+"," "IN" Instruction "END"


Var_instruction_ident := name
--> %name = alloca llvm_type(type(name)) 


Becomes_equal_instruction ::=
Ident+"." [ "(" Term+"," ")" ] ":=" Term
| Ident+"." ":= " Expr_array
| Ident+"." ":=" "bool" "(" Condition ")"

Callup_instruction ::=
[ (Ident+".")+"," "<--" ] Ident+"." [ "(" (Term | String_lit)+"," ")" ]
Sequence_instruction ::= Instruction ";" Instruction

If_instruction ::=
"IF" Condition "THEN" Instruction
( "ELSIF" Condition "THEN" Instruction )* [ "ELSE" Instruction ]
"END"

Case_instruction ::=
"CASE" Simple_term "OF"
"EITHER" Simple_term+"," "THEN" Instruction ( "OR" Simple_term+"," "THEN" Instruction )*
[ "ELSE" Instruction ]
"END"
"END"